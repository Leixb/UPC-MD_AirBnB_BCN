% vim: spell spelllang=en:
%! TEX root = **/main.tex

% Initial working plan (two pages): Including Gantt, division of tasks and
% brief risk contingency plan(see Working team resources slides in the website
% section entitled Working team resources)

\subsection{Gantt}%
\label{sub:gantt}

\subsection{Division of tasks}%
\label{sub:division_of_tasks}

\newcommand*\rot{\rotatebox{90}}
\newcommand*\X{\ding{56}}
\newcommand*\x{\ding{55}}
\begin{table}[H]
\centering
\begin{tabular}{@{}lcccc@{}}
             & \rot{Aleix Boné} & \rot{Eduard Bosch} & \rot{David Gili} & \rot{Albert Mercadé} \\
\toprule
\textbf{Coordination}                           &    & \X &    &    \\ \midrule
\textbf{Motivation and Description}             &    &    & \x &    \\ \midrule
\textbf{Data source presentation}               &    &    &    &    \\
Description of data source                      &    &    &    &    \\ \midrule
\textbf{Formal description of data structure}   &    &    &    &    \\
Row descriptions                                &    &    &    &    \\
Metadata Table                                  &    &    &    &    \\
Scope of study                                  &    &    &    &    \\ \midrule
\textbf{Data mining process performed}          &    &    &    &    \\
workflow                                        &    &    &    &    \\ \midrule
\textbf{Description of prepossessing}           &    &    &    &    \\ \midrule
\textbf{Basic statistical descriptive analysis} &    &    &    &    \\
Univariate of all variables                     &    &    &    &    \\
Bivariate when relevant                         &    &    &    &    \\
Descriptives before \& after                    &    &    &    &    \\
Conclusion describing data                      &    &    &    &    \\ \midrule
\textbf{PCA analysis for numerical variables}   &    &    &    &    \\
Scree plot                                      &    &    &    &    \\
Factorial map visualization                     &    &    &    &    \\
\textbf{Hierarchical Clustering}                &    &    &    &    \\ \midrule
Description of data used                        &    &    &    &    \\
Clustering method used                          &    &    &    &    \\
Resulting Dendogram                             &    &    &    &    \\
Disscussion about number of clusters            &    &    &    &    \\
Table with description of cluster size          &    &    &    &    \\ \midrule
\textbf{Profiling of clusters}                  &    &    &    &    \\
Profiling graphs                                &    &    &    &    \\
Profiling graphs                                &    &    &    &    \\
\\ \bottomrule
\end{tabular}
\end{table}

\subsection{Risk contingency plan}%
\label{sub:risk_contingency_plan}


\begin{table}[H]
\centering
\begin{tabular}{@{}p{5cm}p{5cm}p{5cm}@{}}
\toprule
Risk & How to prevent & How to manage \\ \midrule
A team member leaves the group & All tasks have at least two members assigned & Pending work reassigned to balance the workload\\
\addlinespace[0.5em]
A team member couldn't complete his assigned tasks one week & All tasks have at least two members assigned & Upon being notified that one member can't fulfill his tasks, the members that share tasks with that person will exceptionally complete his tasks for that week\\
\addlinespace[0.5em]
The project (data \& R scripts) is lost or corrupted & Have multiple back up copies of GitHub repository both on the cloud (Overleaf \& GitLab) and locally in our computers & Recover the whole project from one of the other back ups\\
\bottomrule
\end{tabular}
\end{table}

%
% A fairly complicated example from section 2.9 of the package
% documentation. This reproduces an example from Wikipedia:
% http://en.wikipedia.org/wiki/Gantt_chart
%
\definecolor{barblue}{RGB}{153,204,254}
\definecolor{groupblue}{RGB}{51,102,254}
\definecolor{linkred}{RGB}{165,0,33}
\renewcommand\sfdefault{phv}
\renewcommand\mddefault{mc}
\renewcommand\bfdefault{bc}
\setganttlinklabel{s-s}{START-TO-START}
\setganttlinklabel{f-s}{FINISH-TO-START}
\setganttlinklabel{f-f}{FINISH-TO-FINISH}
\sffamily
\begin{ganttchart}[
    canvas/.append style={fill=none, draw=black!5, line width=.75pt},
    hgrid style/.style={draw=black!5, line width=.75pt},
    vgrid={*1{draw=black!5, line width=.75pt}},
    today=7,
    today rule/.style={
        draw=black!64,
        dash pattern=on 3.5pt off 4.5pt,
        line width=1.5pt
    },
    today label font=\small\bfseries,
    title/.style={draw=none, fill=none},
    title label font=\bfseries\footnotesize,
    title label node/.append style={below=7pt},
    include title in canvas=false,
    bar label font=\mdseries\small\color{black!70},
    bar label node/.append style={left=2cm},
    bar/.append style={draw=none, fill=black!63},
    bar incomplete/.append style={fill=barblue},
    bar progress label font=\mdseries\footnotesize\color{black!70},
    group incomplete/.append style={fill=groupblue},
    group left shift=0,
    group right shift=0,
    group height=.5,
    group peaks tip position=0,
    group label node/.append style={left=.6cm},
    group progress label font=\bfseries\small,
    link/.style={-latex, line width=1.5pt, linkred},
    link label font=\scriptsize\bfseries,
    link label node/.append style={below left=-2pt and 0pt}
    ]{1}{13}
    \gantttitle[
    title label node/.append style={below left=7pt and -3pt}
    ]{WEEKS:\quad1}{1}
    \gantttitlelist{2,...,13}{1} \\
    \ganttgroup[progress=57]{WBS 1 Summary Element 1}{1}{10} \\
    \ganttbar[
    progress=75,
    name=WBS1A
    ]{\textbf{WBS 1.1} Activity A}{1}{8} \\
    \ganttbar[
    progress=67,
    name=WBS1B
    ]{\textbf{WBS 1.2} Activity B}{1}{3} \\
    \ganttbar[
    progress=50,
    name=WBS1C
    ]{\textbf{WBS 1.3} Activity C}{4}{10} \\
    \ganttbar[
    progress=0,
    name=WBS1D
    ]{\textbf{WBS 1.4} Activity D}{4}{10} \\[grid]
    \ganttgroup[progress=0]{WBS 2 Summary Element 2}{4}{10} \\
    \ganttbar[progress=0]{\textbf{WBS 2.1} Activity E}{4}{5} \\
    \ganttbar[progress=0]{\textbf{WBS 2.2} Activity F}{6}{8} \\
    \ganttbar[progress=0]{\textbf{WBS 2.3} Activity G}{9}{10}
    \ganttlink[link type=s-s]{WBS1A}{WBS1B}
    \ganttlink[link type=f-s]{WBS1B}{WBS1C}
    \ganttlink[
    link type=f-f,
    link label node/.append style=left
    ]{WBS1C}{WBS1D}
\end{ganttchart}

%
% A simpler example from the package documentation:
%
\begin{ganttchart}{1}{12}
    \gantttitle{2011}{12} \\
    \gantttitlelist{1,...,12}{1} \\
    \ganttgroup{Group 1}{1}{7} \\
    \ganttbar{Task 1}{1}{2} \\
    \ganttlinkedbar{Task 2}{3}{7} \ganttnewline
    \ganttmilestone{Milestone}{7} \ganttnewline
    \ganttbar{Final Task}{8}{12}
    \ganttlink{elem2}{elem3}
    \ganttlink{elem3}{elem4}
\end{ganttchart}
