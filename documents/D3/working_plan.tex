% vim: spell spelllang=en:
%! TEX root = **/main.tex

% Initial working plan (two pages): Including Gantt, division of tasks and
% brief risk contingency plan(see Working team resources slides in the website
% section entitled Working team resources)

\subsection{Gantt}%
\label{sub:gantt}

\subsection{Division of tasks}%
\label{sub:division_of_tasks}

\newcommand*\rot{\rotatebox{90}}
\newcommand*\X{\ding{56}}
\newcommand*\x{\ding{55}}
\begin{table}[H]
\centering
\begin{tabular}{@{}l|c|c|c|c@{}}
             & \rot{Aleix Boné} & \rot{Eduard Bosch} & \rot{David Gili} & \rot{Albert Mercadé} \\
\toprule
\textbf{Coordination}                           &    &  &\X    &    \\ \midrule
\textbf{Motivation and Description}             & \x   &    & & \X   \\ \midrule
\textbf{Data source presentation}               &    &    &     &    \\
Description of data source                      &    & \x   &    & \X  \\ \midrule
\textbf{Formal description of data structure}   &    &    &    &    \\
Row descriptions                                &    &    &\X    & \x   \\
Metadata Table                                  &\x    &\x    & \X   &    \\
Scope of study                                  &\X    &\x    &    &    \\ \midrule
%\textbf{Data mining process performed}          &\X    &    &\x    &    \\
%Workflow                                        &    &    &    &    \\ \midrule
\textbf{Description of prepossessing}           &    &\X    &    &x    \\ \midrule
\textbf{Basic statistical descriptive analysis} &    &    &    &    \\
Univariate of all variables                     &    &\x    &    & \X \\
Bivariate when relevant                         &    & \x   &\X    &    \\
Descriptives before \& after                    &    &    &\X    &\x  \\
Conclusion describing data                      &\X  & \x &    &    \\ \midrule
\textbf{PCA analysis for numerical variables}   &   &    &  &    \\
Scree plot                                      &    &    &\x  & \X  \\
Factorial map visualization                     &    & \X & \x &    \\ \midrule
\textbf{Hierarchical Clustering}                &  &    &    &  \\ 
Description of data used                        &\x  & \X &    &    \\
Clustering method used                          & \X &\x  &    &    \\
Resulting Dendogram                             &    &    &\X   &\x  \\
Discussion about number of clusters             &\x  &\X  &    &    \\
Table with description of cluster size          &\x  &    &\X  &    \\ \midrule
\textbf{Profiling of clusters}                  &    &    &    & \\ 
Profiling graphs                                &    &    &\X  &\x  \\
Profiling graphs                                &    &\x  &    & \X \\ \bottomrule
\end{tabular}
\end{table}

\subsection{Risk contingency plan}%
\label{sub:risk_contingency_plan}


\begin{table}[H]
\centering
\begin{tabular}{@{}p{5cm}p{5cm}p{5cm}@{}}
\toprule
Risk & How to prevent & How to manage \\ \midrule
A team member leaves the group & All tasks have at least two members assigned & Pending work reassigned to balance the workload\\
\addlinespace[0.5em]
A team member couldn't complete his assigned tasks one week & All tasks have at least two members assigned & Upon being notified that one member can't fulfill his tasks, the members that share tasks with that person will exceptionally complete his tasks for that week.\\
\addlinespace[0.5em]
The project (data \& R scripts) is lost or corrupted & Have multiple back up copies of GitHub repository both on the cloud (Overleaf \& GitLab) and locally in our computers & Recover the whole project from one of the other back ups.\\
\addlinespace[0.5em]
A member does not have time due to work, an exam or a project & Each member has to make the group aware of busy weeks so the group can schedule the tasks accordingly & If an unexpected time-consuming problem occurs communicate with the group immediately so that work can be redistributed in advance.\\
\addlinespace[0.5em]
Conflicting opinions between group members on a specific topic & Stick whenever possible to what we agreed on the preliminary plan & If there are conflicting opinions in a specific decision let each member explain his point of view. After that vote and proceed with the majority opinion.\\
\addlinespace[0.5em]
We get inconclusive or conflicting results in some step & Try to finish each step days before delivery day so there is time to solve the problem & As the project requires progressive work, solve the problem in this each before starting the next one. If the problem can't be pinpointed communicate with our lab teacher.  
\\
\addlinespace[0.5em]
The Government imposes a quarantine and we aren't allowed to do face-to-face classes or meetings & There's nothing in our hands to prevent this & Set up a Slack to coordinate the work and a Discord group to do online group meetings.\\

\bottomrule
\end{tabular}
\end{table}

\newgeometry{top=0.5cm, bottom=0.5cm, left=0.5cm, right=0.5cm}
\begin{landscape}
\begin{center}
\begin{ganttchart}[
vgrid={*{6}{draw=none}, dotted},
x unit=.55cm,
y unit title=1cm,
y unit chart=1cm,
    time slot format=isodate
    ]{2020-09-14}{2020-10-18}
\gantttitlecalendar{month=name, day}
\ganttnewline

\ganttgroup{Motivation and Description}{2020-09-16}{2020-09-23}
\ganttnewline

\ganttgroup{Data source presentation}{2020-09-16}{2020-09-23}
\ganttnewline

\ganttbar{Description of data source}{2020-09-16}{2020-09-23}
\ganttnewline

\ganttgroup{Formal description of data structure}{2020-09-23}{2020-09-30}
\ganttnewline
\ganttbar{Metadata Table}{2020-09-23}{2020-09-25}
\ganttnewline
\ganttbar{Scope of study}{2020-09-25}{2020-09-30}
\ganttnewline

\ganttgroup{Description of preprocessing}{2020-09-23}{2020-09-27}
\ganttnewline

\ganttgroup{Basic statistical descriptive analysis}{2020-09-27}{2020-10-07}
\ganttnewline
\ganttbar{Univariate analysis}{2020-09-27}{2020-09-30}
\ganttnewline
\ganttbar{Bivariate analysis}{2020-10-01}{2020-10-04}
\ganttnewline
\ganttbar{Conclusion describing data}{2020-10-04}{2020-10-07}
\ganttnewline
\end{ganttchart}
\end{center}

\pagebreak

\begin{center}
\begin{ganttchart}[
vgrid={*{6}{draw=none}, dotted},
x unit=.65cm,
y unit title=1cm,
y unit chart=1cm,
    time slot format=isodate
    ]{2020-10-01}{2020-10-28}
\gantttitlecalendar{month=name, day}
\ganttnewline

\ganttgroup{PCA analysis for numerical variables}{2020-10-07}{2020-10-14}
\ganttnewline
\ganttbar{Scree plot}{2020-10-07}{2020-10-11}
\ganttnewline
\ganttbar{Factorial map visualization}{2020-10-11}{2020-10-14}
\ganttnewline

\ganttgroup{Hierarchical Clustering}{2020-10-14}{2020-10-21}
\ganttnewline
\ganttbar{Description of data used}{2020-10-01}{2020-10-28}
\ganttnewline
\ganttbar{Clustering method used}{2020-10-01}{2020-10-28}
\ganttnewline
\ganttbar{Resulting Dendogram}{2020-10-01}{2020-10-28}
\ganttnewline
\ganttbar{Discussion about number of clusters}{2020-10-01}{2020-10-28}
\ganttnewline
\ganttbar{Table with description of cluster size}{2020-10-01}{2020-10-28}
\ganttnewline

\ganttgroup{Profiling of clusters}{2020-10-21}{2020-10-28}
\ganttnewline
\ganttbar{Profiling graphs}{2020-10-21}{2020-10-28}
\ganttnewline

\end{ganttchart}
\end{center}
\end{landscape}

\restoregeometry
