% vim: spell spelllang=en:
%! TEX root = **/00-main.tex
\documentclass[12pt, oneside]{article}
%\documentclass[draft, 12pt, oneside]{article}
%\documentclass[final, 12pt, oneside]{article}
\usepackage[a4paper, left=2.5cm, right=2.5cm, top=2.5cm, bottom=2.5cm]{geometry}
%\usepackage[showframe, a4paper, left=2.5cm, right=2.5cm, top=2.5cm, bottom=2.5cm]{geometry}

\usepackage[T1]{fontenc}

%% Not needed with lualatex
%\usepackage[utf8]{inputenc} % Use unicode for input
%\usepackage{lmodern}

% With lualatex:
\usepackage{fontspec}
\setmonofont[Scale=MatchLowercase]{DejaVu Sans Mono}

\usepackage{microtype}

\usepackage{polyglossia}
\setdefaultlanguage{english}

\usepackage{hyphenat}

%% Bibliography:
\usepackage[
    backend=biber,
    style=numeric,
]{biblatex}
\DeclareNameAlias{default}{last-first}

%\DeclareQuoteAlias{spanish}{catalan}

\addbibresource{biblio.bib}
%% see:
%% https://www.sharelatex.com/learn/Bibliography_management_in_LaTeX#The_bibliography_file

% For cross references
\usepackage{color, xcolor}
\usepackage[breaklinks, colorlinks = true]{hyperref}
%hyperref configuration so that it doesn't contrast so much colorlinks,
\hypersetup{
   linkcolor={black},
   citecolor={black},
   urlcolor={blue!80!black}
}

\usepackage{mathtools}                    % amsmath + more
\usepackage{amsthm}                       % Theorem enviroment
\usepackage{amssymb}                      % More symbols
\usepackage{amstext}                      % Text inside mathenv

\usepackage{relsize}                      % Bigger math with mathlarger{___}
\usepackage{nicefrac}                     % nice fractions in one line

\usepackage[export]{adjustbox}            % Adjust table size
\usepackage{float}                        % Force tables and images position (H and H!)
\usepackage{wrapfig}                      % Wrap images like in HTML

%\usepackage{tabularx, colortbl, booktabs} % Better tables
%\usepackage{longtable}                    % Multiple page table
\usepackage{xltabular, colortbl, booktabs} % longtable + tabularx (has bug with booktabs: fix below)

% bug booktabs + xltabular
\makeatletter
\def\@BTrule[#1]{%
  \ifx\longtable\undefined
    \let\@BTswitch\@BTnormal
  \else\ifx\hline\LT@hline
    \nobreak
    \let\@BTswitch\@BLTrule
  \else
     \let\@BTswitch\@BTnormal
  \fi\fi
  \global\@thisrulewidth=#1\relax
  \ifnum\@thisruleclass=\tw@\vskip\@aboverulesep\else
  \ifnum\@lastruleclass=\z@\vskip\@aboverulesep\else
  \ifnum\@lastruleclass=\@ne\vskip\doublerulesep\fi\fi\fi
  \@BTswitch}
\makeatother

% Split cell in lines and more formating options inside table
\usepackage{array, multirow, multicol, makecell}

%\usepackage{subcaption}                     % Subfigures
%\usepackage[framemethod=tikz]{mdframed}     % Custom frames

\usepackage[bottom]{footmisc} % Footnotes at bottom of page

%\usepackage[alsoload=hep]{siunitx}          % SI units and uncertainties
%\sisetup{locale = FR}                       % Commas and so on for spanish
%\sisetup{separate-uncertainty=true}
%\sisetup{
  %per-mode=fraction,
  %fraction-function=\nicefrac
%}

%\usepackage{tikz}
%%\usetikzlibrary{arrows}
%%\usetikzlibrary{scopes}
%\usetikzlibrary{babel}

%% Custom Math operators (functions not in italic in math mode):
%\DeclareMathOperator{\cis}{cis}

% Set pages to landscape
\usepackage{pdflscape}
\usepackage{afterpage}

\usepackage{pdfpages}

\usepackage{comment}        % multiline comments

\usepackage{pifont}         % some fancy symbols

\usepackage{xspace}         % to create commands with space at end

%\usepackage{listings}       % For code listings (prefer minted)
\usepackage{minted}         % -shell-escape
\definecolor{codeBg}{HTML}{FFFDE7}
\setminted[r]{
    %style=pastie,
    frame=lines,
    framesep=3mm,
    linenos,
    breaklines=true,
    encoding=utf8,
    fontsize=\footnotesize,
    bgcolor=codeBg
}

\usepackage{pgfgantt}           % Gantt chart

\newcommand{\whitepage}{        % white page without adding to pagenumber
    \clearpage\thispagestyle{empty}\addtocounter{page}{-1} \newpage \clearpage
}
\newcommand{\ts}{\textsuperscript}

\usepackage{tocbibind}

\usepackage[justification=centering]{caption}

\usepackage[english]{cleveref}
\usepackage{csquotes}       % For bibliography quotations

\usepackage{subcaption}
\input{../preamble_rmarkdown}

\usepackage{pgfgantt}

%\graphicspath{{../..}{..}{../../analysis/anal_files/figure-latex}}

\newcommand{\airbnb}{\emph{Airbnb}\xspace}
\newcommand{\NA}{\emph{NA}\xspace}

\title{MD - \airbnb Barcelona Listings}
\author{
    Aleix Bon\'e\\
    Eduard Bosch\\
    David Gili\\
    Albert Mercad\'e\\
}
\date{\today}

\begin{document}


\input{titlepage}

\setcounter{tocdepth}{2}
\tableofcontents

% Initial working plan (two pages): Including Gantt, division of tasks and
% brief risk contingency plan(see Working team resources slides in the website
% section entitled Working team resources)
\clearpage
%\section{Working Plan}
% vim: spell spelllang=en:
%! TEX root = **/main.tex

% Initial working plan (two pages): Including Gantt, division of tasks and
% brief risk contingency plan(see Working team resources slides in the website
% section entitled Working team resources)

\subsection{Gantt}%
\label{sub:gantt}

\subsection{Division of tasks}%
\label{sub:division_of_tasks}

\newcommand*\rot{\rotatebox{90}}
\newcommand*\X{\ding{56}}
\newcommand*\x{\ding{55}}
\begin{table}[H]
\centering
\begin{tabular}{@{}l|c|c|c|c@{}}
             & \rot{Aleix Boné} & \rot{Eduard Bosch} & \rot{David Gili} & \rot{Albert Mercadé} \\
\toprule
\textbf{Coordination}                           &    & \X &    &    \\ \midrule
\textbf{Motivation and Description}             &    &    & \x &    \\ \midrule
\textbf{Data source presentation}               &    &    &    &    \\
Description of data source                      &    &    &    &    \\ \midrule
\textbf{Formal description of data structure}   &    &    &    &    \\
Row descriptions                                &    &    &    &    \\
Metadata Table                                  &    &    &    &    \\
Scope of study                                  &    &    &    &    \\ \midrule
\textbf{Data mining process performed}          &    &    &    &    \\
Workflow                                        &    &    &    &    \\ \midrule
\textbf{Description of prepossessing}           &    &    &    &    \\ \midrule
\textbf{Basic statistical descriptive analysis} &    &    &    &    \\
Univariate of all variables                     &    &    &    &    \\
Bivariate when relevant                         &    &    &    &    \\
Descriptives before \& after                    &    &    &    &    \\
Conclusion describing data                      &    &    &    &    \\ \midrule
\textbf{PCA analysis for numerical variables}   &    &    &    &    \\
Scree plot                                      &    &    &    &    \\
Factorial map visualization                     &    &    &    &    \\
\textbf{Hierarchical Clustering}                &    &    &    &    \\ \midrule
Description of data used                        &    &    &    &    \\
Clustering method used                          &    &    &    &    \\
Resulting Dendogram                             &    &    &    &    \\
Discussion about number of clusters             &    &    &    &    \\
Table with description of cluster size          &    &    &    &    \\ \midrule
\textbf{Profiling of clusters}                  &    &    &    &    \\
Profiling graphs                                &    &    &    &    \\
Profiling graphs                                &    &    &    &    \\ \bottomrule
\end{tabular}
\end{table}

\subsection{Risk contingency plan}%
\label{sub:risk_contingency_plan}


\begin{table}[H]
\centering
\begin{tabular}{@{}p{5cm}p{5cm}p{5cm}@{}}
\toprule
Risk & How to prevent & How to manage \\ \midrule
A team member leaves the group & All tasks have at least two members assigned & Pending work reassigned to balance the workload\\
\addlinespace[0.5em]
A team member couldn't complete his assigned tasks one week & All tasks have at least two members assigned & Upon being notified that one member can't fulfill his tasks, the members that share tasks with that person will exceptionally complete his tasks for that week\\
\addlinespace[0.5em]
The project (data \& R scripts) is lost or corrupted & Have multiple back up copies of GitHub repository both on the cloud (Overleaf \& GitLab) and locally in our computers & Recover the whole project from one of the other back ups\\
\bottomrule
\end{tabular}
\end{table}

\newgeometry{top=0.5cm, bottom=0.5cm, left=0.5cm, right=0.5cm}
\begin{landscape}
\begin{center}
\begin{ganttchart}[
vgrid={*{6}{draw=none}, dotted},
x unit=.65cm,
y unit title=1cm,
y unit chart=1cm,
    time slot format=isodate
    ]{2020-10-01}{2020-10-28}
\gantttitlecalendar{month=name, day}

\ganttnewline
\ganttgroup{Coordination}{2020-10-01}{2020-10-28}
\ganttnewline

\ganttgroup{Motivation and Description}{2020-10-01}{2020-10-28}
\ganttnewline

\ganttgroup{Data source presentation}{2020-10-01}{2020-10-28}
\ganttnewline

\ganttbar{Description of data source}{2020-10-01}{2020-10-28}
\ganttnewline

\ganttgroup{Formal description of data structure}{2020-10-01}{2020-10-28}
\ganttnewline
\ganttbar{Row descriptions}{2020-10-01}{2020-10-28}
\ganttnewline
\ganttbar{Metadata Table}{2020-10-01}{2020-10-28}
\ganttnewline
\ganttbar{Scope of study}{2020-10-01}{2020-10-28}
\ganttnewline

\ganttgroup{Data mining process performed}{2020-10-01}{2020-10-28}
\ganttnewline
\ganttbar{workflow}{2020-10-01}{2020-10-28}
\ganttnewline

\ganttgroup{Description of prepossessing}{2020-10-01}{2020-10-28}
\ganttnewline

\ganttgroup{Basic statistical descriptive analysis}{2020-10-01}{2020-10-28}
\ganttnewline
\ganttbar{Univariate of all variables}{2020-10-01}{2020-10-28}
\ganttnewline
\ganttbar{Bivariate when relevant}{2020-10-01}{2020-10-28}
\ganttnewline
\ganttbar{Descriptives before \& after}{2020-10-01}{2020-10-28}
\ganttnewline
\ganttbar{Conclusion describing data}{2020-10-01}{2020-10-28}
\ganttnewline
\end{ganttchart}
\end{center}

\pagebreak

\begin{center}
\begin{ganttchart}[
vgrid={*{6}{draw=none}, dotted},
x unit=.65cm,
y unit title=1cm,
y unit chart=1cm,
    time slot format=isodate
    ]{2020-10-01}{2020-10-28}
\gantttitlecalendar{month=name, day}
\ganttnewline

\ganttgroup{PCA analysis for numerical variables}{2020-10-01}{2020-10-28}
\ganttnewline
\ganttbar{Scree plot}{2020-10-01}{2020-10-28}
\ganttnewline
\ganttbar{Factorial map visualization}{2020-10-01}{2020-10-28}
\ganttnewline

\ganttgroup{Hierarchical Clustering}{2020-10-01}{2020-10-28}
\ganttnewline
\ganttbar{Description of data used}{2020-10-01}{2020-10-28}
\ganttnewline
\ganttbar{Clustering method used}{2020-10-01}{2020-10-28}
\ganttnewline
\ganttbar{Resulting Dendogram}{2020-10-01}{2020-10-28}
\ganttnewline
\ganttbar{Discussion about number of clusters}{2020-10-01}{2020-10-28}
\ganttnewline
\ganttbar{Table with description of cluster size}{2020-10-01}{2020-10-28}
\ganttnewline

\ganttgroup{Profiling of clusters}{2020-10-01}{2020-10-28}
\ganttnewline
\ganttbar{Profiling graphs}{2020-10-01}{2020-10-28}
\ganttnewline

\end{ganttchart}
\end{center}
\end{landscape}

\restoregeometry


% Metadata file describing the selection of variables considered for the
% analysis (see slide nr 8 in Data and Metadata slides from Theme 2. Data
% Preparation)
\clearpage
%\begin{landscape}
%\section{Metadata file}
\includepdf[pages=-,landscape=true]{metadata}
%% vim: spell spelllang=en:
%! TEX root = **/main.tex

% Metadata file describing the selection of variables considered for the
% analysis (see slide nr 8 in Data and Metadata slides from Theme 2. Data
% Preparation)

\begin{itemize}
         \item \textbf{url:}
         \item \textbf{Inclusioncriteria:}
         \item \textbf{n:}
         \item \textbf{K:}
\end{itemize}

\begin{table}[H]
         \centering
\begin{tabular}{@{}lllllllll@{}}
\toprule
Variable & Modalities & meaning & Type & \makecell{Measuring\\ unit} & \makecell{Missing\\ code} & \makecell{Measuring\\ procedure} & Range & Role \\ \midrule
         &            &         &      &                &              &                     &       &      \\ \bottomrule
\end{tabular}
\end{table}

%\end{landscape}

% Basic initial univariate descriptive statistics of raw variables (see
% provided R scripts for automatic descriptive analysis in  Lab Session2.
% Preprocessing (II). Markdow produces word files, the R script is interpreted.
% The scripts ARE ORIENTATIVE. Modify whenever required)
\clearpage
\section{Descriptive Statistics of raw variables}%
\label{sec:descriptive_analisis}
% vim: spell spelllang=en:
%! TEX root = **/main.tex

% Basic initial univariate descriptive statistics of raw variables (see
% provided R scripts for automatic descriptive analysis in  Lab Session2.
% Preprocessing (II). Markdow produces word files, the R script is interpreted.
% The scripts ARE ORIENTATIVE. Modify whenever required)

\#install.packages(``rmarkdown'') \#library(``rmarkdown'')

\begin{Shaded}
\begin{Highlighting}[]
\NormalTok{data \textless{}{-}}\StringTok{ }\KeywordTok{read.csv}\NormalTok{(}\KeywordTok{gzfile}\NormalTok{(}\StringTok{"../listings.csv.gz"}\NormalTok{))}

\KeywordTok{summary}\NormalTok{(data)}
\end{Highlighting}
\end{Shaded}

\begin{verbatim}
##        id           listing_url          scrape_id        last_scraped      
##  Min.   :   21974   Length:20703       Min.   :2.02e+13   Length:20703      
##  1st Qu.:14737064   Class :character   1st Qu.:2.02e+13   Class :character  
##  Median :27479730   Mode  :character   Median :2.02e+13   Mode  :character  
##  Mean   :25883290                      Mean   :2.02e+13                     
##  3rd Qu.:38872258                      3rd Qu.:2.02e+13                     
##  Max.   :45086940                      Max.   :2.02e+13                     
##                                                                             
##      name           description        neighborhood_overview picture_url       
##  Length:20703       Length:20703       Length:20703          Length:20703      
##  Class :character   Class :character   Class :character      Class :character  
##  Mode  :character   Mode  :character   Mode  :character      Mode  :character  
##                                                                                
##                                                                                
##                                                                                
##                                                                                
##     host_id            host_url          host_name          host_since       
##  Min.   :     3073   Length:20703       Length:20703       Length:20703      
##  1st Qu.:  9364267   Class :character   Class :character   Class :character  
##  Median : 59065387   Mode  :character   Mode  :character   Mode  :character  
##  Mean   :110213147                                                           
##  3rd Qu.:197962361                                                           
##  Max.   :363926491                                                           
##                                                                              
##  host_location       host_about        host_response_time host_response_rate
##  Length:20703       Length:20703       Length:20703       Length:20703      
##  Class :character   Class :character   Class :character   Class :character  
##  Mode  :character   Mode  :character   Mode  :character   Mode  :character  
##                                                                             
##                                                                             
##                                                                             
##                                                                             
##  host_acceptance_rate host_is_superhost  host_thumbnail_url host_picture_url  
##  Length:20703         Length:20703       Length:20703       Length:20703      
##  Class :character     Class :character   Class :character   Class :character  
##  Mode  :character     Mode  :character   Mode  :character   Mode  :character  
##                                                                               
##                                                                               
##                                                                               
##                                                                               
##  host_neighbourhood host_listings_count host_total_listings_count
##  Length:20703       Min.   :  0.00      Min.   :  0.00           
##  Class :character   1st Qu.:  1.00      1st Qu.:  1.00           
##  Mode  :character   Median :  2.00      Median :  2.00           
##                     Mean   : 16.81      Mean   : 16.81           
##                     3rd Qu.: 11.00      3rd Qu.: 11.00           
##                     Max.   :551.00      Max.   :551.00           
##                     NA's   :8           NA's   :8                
##  host_verifications host_has_profile_pic host_identity_verified
##  Length:20703       Length:20703         Length:20703          
##  Class :character   Class :character     Class :character      
##  Mode  :character   Mode  :character     Mode  :character      
##                                                                
##                                                                
##                                                                
##                                                                
##  neighbourhood      neighbourhood_cleansed neighbourhood_group_cleansed
##  Length:20703       Length:20703           Length:20703                
##  Class :character   Class :character       Class :character            
##  Mode  :character   Mode  :character       Mode  :character            
##                                                                        
##                                                                        
##                                                                        
##                                                                        
##     latitude       longitude     property_type       room_type        
##  Min.   :41.35   Min.   :2.086   Length:20703       Length:20703      
##  1st Qu.:41.38   1st Qu.:2.157   Class :character   Class :character  
##  Median :41.39   Median :2.168   Mode  :character   Mode  :character  
##  Mean   :41.39   Mean   :2.167                                        
##  3rd Qu.:41.40   3rd Qu.:2.178                                        
##  Max.   :41.46   Max.   :2.229                                        
##                                                                       
##   accommodates    bathrooms      bathrooms_text        bedrooms     
##  Min.   : 1.000   Mode:logical   Length:20703       Min.   : 1.000  
##  1st Qu.: 2.000   NA's:20703     Class :character   1st Qu.: 1.000  
##  Median : 2.000                  Mode  :character   Median : 1.000  
##  Mean   : 3.297                                     Mean   : 1.604  
##  3rd Qu.: 4.000                                     3rd Qu.: 2.000  
##  Max.   :16.000                                     Max.   :16.000  
##                                                     NA's   :684     
##       beds         amenities            price           minimum_nights   
##  Min.   : 0.000   Length:20703       Length:20703       Min.   :   1.00  
##  1st Qu.: 1.000   Class :character   Class :character   1st Qu.:   1.00  
##  Median : 2.000   Mode  :character   Mode  :character   Median :   2.00  
##  Mean   : 2.233                                         Mean   :  10.39  
##  3rd Qu.: 3.000                                         3rd Qu.:   7.00  
##  Max.   :48.000                                         Max.   :1124.00  
##  NA's   :409                                                             
##  maximum_nights      minimum_minimum_nights maximum_minimum_nights
##  Min.   :1.000e+00   Min.   :   1.00        Min.   :   1.00       
##  1st Qu.:9.000e+01   1st Qu.:   1.00        1st Qu.:   2.00       
##  Median :1.125e+03   Median :   2.00        Median :   3.00       
##  Mean   :2.082e+05   Mean   :  10.16        Mean   :  11.38       
##  3rd Qu.:1.125e+03   3rd Qu.:   6.00        3rd Qu.:  20.00       
##  Max.   :2.147e+09   Max.   :1123.00        Max.   :1124.00       
##                                                                   
##  minimum_maximum_nights maximum_maximum_nights minimum_nights_avg_ntm
##  Min.   :1.000e+00      Min.   :1.000e+00      Min.   :   1.00       
##  1st Qu.:3.300e+02      1st Qu.:3.300e+02      1st Qu.:   1.40       
##  Median :1.125e+03      Median :1.125e+03      Median :   2.70       
##  Mean   :2.083e+05      Mean   :3.120e+05      Mean   :  10.67       
##  3rd Qu.:1.125e+03      3rd Qu.:1.125e+03      3rd Qu.:  12.10       
##  Max.   :2.147e+09      Max.   :2.147e+09      Max.   :1123.00       
##                                                                      
##  maximum_nights_avg_ntm calendar_updated has_availability   availability_30
##  Min.   :1.000e+00      Mode:logical     Length:20703       Min.   : 0.00  
##  1st Qu.:3.300e+02      NA's:20703       Class :character   1st Qu.: 0.00  
##  Median :1.125e+03                       Mode  :character   Median :24.00  
##  Mean   :3.118e+05                                          Mean   :17.55  
##  3rd Qu.:1.125e+03                                          3rd Qu.:30.00  
##  Max.   :2.147e+09                                          Max.   :30.00  
##                                                                            
##  availability_60 availability_90 availability_365 calendar_last_scraped
##  Min.   : 0.00   Min.   : 0.00   Min.   :  0.0    Length:20703         
##  1st Qu.: 0.00   1st Qu.: 3.00   1st Qu.: 55.0    Class :character     
##  Median :52.00   Median :80.00   Median :180.0    Mode  :character     
##  Mean   :36.43   Mean   :56.01   Mean   :191.3                         
##  3rd Qu.:59.00   3rd Qu.:89.00   3rd Qu.:352.0                         
##  Max.   :60.00   Max.   :90.00   Max.   :365.0                         
##                                                                        
##  number_of_reviews number_of_reviews_ltm number_of_reviews_l30d
##  Min.   :  0.0     Min.   :  0.000       Min.   : 0.0000       
##  1st Qu.:  0.0     1st Qu.:  0.000       1st Qu.: 0.0000       
##  Median :  5.0     Median :  1.000       Median : 0.0000       
##  Mean   : 33.1     Mean   :  6.401       Mean   : 0.1621       
##  3rd Qu.: 36.0     3rd Qu.:  9.000       3rd Qu.: 0.0000       
##  Max.   :743.0     Max.   :278.000       Max.   :15.0000       
##                                                                
##  first_review       last_review        review_scores_rating
##  Length:20703       Length:20703       Min.   : 20.00      
##  Class :character   Class :character   1st Qu.: 88.00      
##  Mode  :character   Mode  :character   Median : 93.00      
##                                        Mean   : 91.08      
##                                        3rd Qu.: 98.00      
##                                        Max.   :100.00      
##                                        NA's   :5971        
##  review_scores_accuracy review_scores_cleanliness review_scores_checkin
##  Min.   : 2.000         Min.   : 2.000            Min.   : 2.000       
##  1st Qu.: 9.000         1st Qu.: 9.000            1st Qu.: 9.000       
##  Median :10.000         Median : 9.000            Median :10.000       
##  Mean   : 9.379         Mean   : 9.227            Mean   : 9.558       
##  3rd Qu.:10.000         3rd Qu.:10.000            3rd Qu.:10.000       
##  Max.   :10.000         Max.   :10.000            Max.   :10.000       
##  NA's   :5982           NA's   :5980              NA's   :5986         
##  review_scores_communication review_scores_location review_scores_value
##  Min.   : 2.000              Min.   : 2.000         Min.   : 2.000     
##  1st Qu.: 9.000              1st Qu.: 9.000         1st Qu.: 9.000     
##  Median :10.000              Median :10.000         Median : 9.000     
##  Mean   : 9.551              Mean   : 9.618         Mean   : 9.055     
##  3rd Qu.:10.000              3rd Qu.:10.000         3rd Qu.:10.000     
##  Max.   :10.000              Max.   :10.000         Max.   :10.000     
##  NA's   :5978                NA's   :5985           NA's   :5985       
##    license          instant_bookable   calculated_host_listings_count
##  Length:20703       Length:20703       Min.   :  1.00                
##  Class :character   Class :character   1st Qu.:  1.00                
##  Mode  :character   Mode  :character   Median :  2.00                
##                                        Mean   : 14.03                
##                                        3rd Qu.: 11.00                
##                                        Max.   :169.00                
##                                                                      
##  calculated_host_listings_count_entire_homes
##  Min.   :  0.00                             
##  1st Qu.:  0.00                             
##  Median :  1.00                             
##  Mean   : 11.41                             
##  3rd Qu.:  7.00                             
##  Max.   :169.00                             
##                                             
##  calculated_host_listings_count_private_rooms
##  Min.   : 0.000                              
##  1st Qu.: 0.000                              
##  Median : 1.000                              
##  Mean   : 2.336                              
##  3rd Qu.: 2.000                              
##  Max.   :97.000                              
##                                              
##  calculated_host_listings_count_shared_rooms reviews_per_month
##  Min.   : 0.0000                             Min.   : 0.010   
##  1st Qu.: 0.0000                             1st Qu.: 0.210   
##  Median : 0.0000                             Median : 0.710   
##  Mean   : 0.0768                             Mean   : 1.179   
##  3rd Qu.: 0.0000                             3rd Qu.: 1.770   
##  Max.   :13.0000                             Max.   :21.410   
##                                              NA's   :5731
\end{verbatim}



% Enumerate which steps of the preprocessing process are used with your
% particular data (consider the steps proposed in slide 4 of Preprocessing
% Slides in Theme 2. Data Preparation. Remember that you have the whole
% complete information on preprocessing in the reference text from the
% linkSurvey of preprocessing. Reference paper (MIE 2001) from complementary
% materials section of Theme 2. Data Preparation)

% List and justify all decisions taken for each preprocessing step

% Additional descriptive statistics of variables that have been modified or
% created by preprocessing
\clearpage
\section{Preprocessing}
% vim: spell spelllang=en:
%! TEX root = **/main.tex

\subsection{Steps and Decisions}%
\label{sub:steps-decisions}

\subsubsection{Formatting issues, building software context}
We haven't encountered any formatting issues or problems with R
not recognising any rows, columns or variable types.

\subsubsection{Determining working matrix}
We discarded many variables from the original data set that didn't 
provide any useful information to our project such as URLs and ids or that couldn't be 
categorised or were uninteresting such as the host\_name, name 
and description of the listing, etc. Furthermore, some columns were duplicated with
different variable names, such as host\_listings\_count and 
host\_total\_listings\_count, we removed those redundancies as well.

\subsubsection{Identification and treatment of missing data}
\subsubsection{Identification and treatment of outliers}
\subsubsection{Identification and treatment of errors}
\subsubsection{Feature selection/extraction, dimensional reduction}
\subsubsection{Instance selection}
\subsubsection{Data transformation}
\subsubsection{Derivation of new variables}
We derived some new variables from the data set in order to flourish new categorical variables.

First, starting with the host\_since variable, which gave us the date the host signed up to AirBnB
to list his property, we extracted two new variables: host\_since\_year and host\_since\_season,
both categorical. host\_since\_year categorizes the host sign up dates into their particular year and since . 
The variable host\_since\_season represents the season of the year during which
the host signed up. For the latter, the modalities are Winter, Spring, Summer and Autumn.

We have also categorized host\_response\_rate and host\_acceptance\_rate
as we converted them from a numerical variables measured in percentages to 
categorical ones with the following modalities: very low, low, average, 
high and very high. The 

These were all the variable derivations we carried out in our data set for this 
project. They were all motivated by the necessity to have more categorical
variables in our data set, as it had plenty numerical and binary variables
but was short of categorical ones.

% Enumerate which steps of the preprocessing process are used with your
% particular data (consider the steps proposed in slide 4 of Preprocessing
% Slides in Theme 2. Data Preparation. Remember that you have the whole
% complete information on preprocessing in the reference text from the
% linkSurvey of preprocessing. Reference paper (MIE 2001) from complementary
% materials section of Theme 2. Data Preparation)

% List and justify all decisions taken for each preprocessing step

% Additional descriptive statistics of variables that have been modified or
% created by preprocessing


\end{document}
