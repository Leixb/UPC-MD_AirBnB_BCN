% vim: spell spelllang=en:
%! TEX root = **/main.tex

\subsection{Steps and Decisions}%
\label{sub:steps-decisions}

\subsubsection{Formatting issues, building software context}
We haven't encountered any formatting issues or problems with R
not recognising any rows, columns or variable types.

\subsubsection{Determining working matrix}
We discarded many variables from the original data set that didn't 
provide any useful information to our project such as URLs and ids or that couldn't be 
categorised or were uninteresting such as the host\_name, name 
and description of the listing, etc. Furthermore, some columns were duplicated with
different variable names, such as host\_listings\_count and 
host\_total\_listings\_count, we removed those redundancies as well.

\subsubsection{Identification and treatment of missing data}
\subsubsection{Identification and treatment of outliers}
\subsubsection{Identification and treatment of errors}
\subsubsection{Feature selection/extraction, dimensional reduction}
\subsubsection{Instance selection}
\subsubsection{Data transformation}
\subsubsection{Derivation of new variables}
We derived some new variables from the data set in order to flourish new categorical variables.

First, starting with the host\_since variable, which gave us the date the host signed up to AirBnB
to list his property, we extracted two new variables: host\_since\_year and host\_since\_season,
both categorical. host\_since\_year categorizes the host sign up dates into their particular year and since . 
The variable host\_since\_season represents the season of the year during which
the host signed up. For the latter, the modalities are Winter, Spring, Summer and Autumn.

We have also categorized host\_response\_rate and host\_acceptance\_rate
as we converted them from a numerical variables measured in percentages to 
categorical ones with the following modalities: very low, low, average, 
high and very high. The 

These were all the variable derivations we carried out in our data set for this 
project. They were all motivated by the necessity to have more categorical
variables in our data set, as it had plenty numerical and binary variables
but was short of categorical ones.

% Enumerate which steps of the preprocessing process are used with your
% particular data (consider the steps proposed in slide 4 of Preprocessing
% Slides in Theme 2. Data Preparation. Remember that you have the whole
% complete information on preprocessing in the reference text from the
% linkSurvey of preprocessing. Reference paper (MIE 2001) from complementary
% materials section of Theme 2. Data Preparation)

% List and justify all decisions taken for each preprocessing step

% Additional descriptive statistics of variables that have been modified or
% created by preprocessing
