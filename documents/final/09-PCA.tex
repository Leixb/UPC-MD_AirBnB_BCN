% vim: spell spelllang=en:
%! TEX root = **/00-main.tex

% PCA analysis for numerical variables:

\section{PCA analysis for numerical variables}%
\label{sec:pca_analysis_for_numerical_variables}

% Scree plot. Specify how many principal components are selected
\subsection{Scree plot}%
\label{sub:scree_plot}


\begin{figure}[H]
    \centering
    \includegraphics[width=0.7\linewidth]{pca_fact-screeplot} % PCA-inertia_cum
    \caption{PCA inertia}%
    \label{fig:pca_inertia}
\end{figure}

\begin{figure}[H]
    \centering
    \includegraphics[width=0.7\linewidth]{PCA-inertia_cum}
    \caption{PCA accumulated inertia}%
    \label{fig:pca_inertia_cum}
\end{figure}

We used the inertia plots to decide the number of factorial axis to analyse. In
\cref{fig:pca_inertia} we can conclude the 4 first axis represent a much
larger amount of variance compared to the others. When looking at
\cref{fig:pca_inertia_cum} we can see that the first 4 axis already contain 63\%
of the variance. If we analyse further with the elbow method we can clearly see
that the slop starts to decrease at around the 4\ts{th} axis. Therefore we decided to
only further analyse the first 4 factorial axis.

% 20.24968405 17.88042469 13.04514781 12.39547495
% 20.24968  38.13011  51.17526  63.57073     %  69.09301  74.04671  78.51992  82.73880


% Factorial map visualisation:
\subsection{Factorial map visualisation}%
\label{sub:factorial_map_visualisation}

% #2 -> caption #3 -> file, #1 -> page
\newcommand{\factorialmap}[2]{
    \begin{figure}[H]
        \centering
        %\includegraphics[trim=3cm 0 3cm 0, clip, width=0.65\linewidth]{plane_#1_#2-var}
        %\parbox[t]{0.24\linewidth}{lorem ipsum dolor sit amet adescipng susisf jsdasj fsdlk ks aksjldsdf jkls}
        \includegraphics[height=0.93\textheight]{pca_fact-plane_#1_#2-var}
        \caption{PCA plane #1 vs #2}%
        \label{fig:plane_#1-#2}
    \end{figure}
}

\newcommand{\contrib}[2]{
    \begin{figure}[H]
        \centering
        \includegraphics[width=0.85\linewidth]{pca_fact-plane_#1_#2-contrib}
        \caption{PCA variable contributions of plane #1 vs #2}%
        \label{fig:contrib_plane_#1-#2}
    \end{figure}
}

\newcommand{\categorica}[4]{
    \begin{figure}[H]
        \centering
        \includegraphics[width=0.85\linewidth]{pca_fact-#3-plane_#1_#2}
        \caption{PCA variable contributions of #4 in plane #1 vs #2}%
        \label{fig:cat-#3-plane-#1-#2}
    \end{figure}
}

\begin{landscape}

%TODO

\factorialmap{1}{2}
\factorialmap{1}{3}
\factorialmap{1}{4}
\factorialmap{2}{3}
\factorialmap{2}{4}
\factorialmap{3}{4}

\contrib{1}{2}
\contrib{3}{4}

\categorica{1}{2}{room_type}{room type}
\categorica{1}{3}{room_type}{room type}
\categorica{1}{4}{room_type}{room type}

\end{landscape}

% (Be sure you use a single landscape pager for each single map in order to
% guarantee visibility of materials to the readers)

% For each factorial map provide:

% - Individuals projections

% - Common projection of numerical variables and modalities of qualitative
% - variables (take care to use correct color codes as explained along the course)

% - Interpretation of relationships among variables observed. When possible,
% - interpret the latent variable associated with the principal axis

% - Conclusions

% Note: All factorial maps must be placed in a single landscape page that makes
% it visible
