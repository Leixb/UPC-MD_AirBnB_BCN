% vim: spell spelllang=en:
%! TEX root = **/00-main.tex

\section{Profiling of clusters}%
\label{sec:profiling_of_clusters}

\newcommand{\profiling}[3]{
\begin{figure}[H]
    \centering
    \includegraphics[width=0.7\textwidth]{prof-c3-#1-#2}
    \caption{#3}%
    \label{fig:prof-#1-#2}
\end{figure}
}

% --- cat

\profiling{host_is_superhost}{side}{Superhost distribution}
\profiling{host_since_year}{percent}{Host since year distribution}
In \Cref{fig:prof-host_is_superhost-side} we can see how cluster 2 has a significantly higher proportional amount of superhosts when comparing to the rest of cluster, while cluster 3 has very few of them. This could indicate that hosts in cluster 2 are better hosts, with more experience and a pretty good record of dealing with clients. The hypothesis is confirmed by \Cref{fig:prof-host_since_year-percent}, where, comparing the same two clusters we see a tendency: cluster 2 is composed mostly of veteran hosts whereas cluster 3 is made up of more novice ones.
%\profiling{host_is_superhost}{stack}{}


%\profiling{host_response_time}{}{} %?



%profiling{room_type}{percent}{room type bar}
\profiling{room_type}{side}{room type bar}
%\profiling{room_type}{stack}{room type bar}


% --- num

\profiling{accommodates}{meanp}{Accommodates}
\profiling{accommodates}{vi}{Accommodates}
%\profiling{accommodates}{bp}{Accommodates}
In \Cref{fig:prof-accommodates-meanp,prof-accommodates-vi} we can see that cluster 4 acommodates 
much more people than the clusters. We can hypothesize that cluster 4 might correspond to listings of big entire houses. 

\profiling{host_listings_count}{meanp}{Host listings count}
\profiling{host_listings_count}{vi}{Host listings count}
\profiling{host_listings_count}{bp}{Host listings count}

In \Cref{fig:host_listings_count,host_listings_count} we can differences in the clusters.
Cluster 4 has the highest mean, followed by cluster 3. The other two seem to be quite 
similar. An hypothesis we have is that cluster 4 might be older listings or ones with a lower minimum night average.

\profiling{price}{meanp}{Price}
\profiling{price}{vi}{Price}
\profiling{price}{bp}{Price}


\subsection{Review scores rating}%
\label{sub:prof-review_scores_rating}

\profiling{review_scores_rating}{meanp}{Review scores rating}
%\profiling{review_scores_rating}{vi}{Review scores rating}
\profiling{review_scores_rating}{bp}{Review scores rating}

Looking at \cref{fig:prof-review_scores_rating-meanp,fig:prof-review_scores_rating-bp} we can see that cluster 3 has a much lower mean of review score rantings than the others. The others seem to have a really similar mean and variance. We can conclude that cluster 3 contains the bad reviews while high reviews seem to distribute between the others.


\profiling{reviews_per_month}{meanp}{Reviews per month}
\profiling{reviews_per_month}{vi}{Reviews per month}
\profiling{reviews_per_month}{bp}{Reviews per month}

\Cref{fig:prof-reviews_per_month-bp,prof-reviews_per_month-meanp} we can see that cluster 2 has higher mean that the rest and
cluster 3 has the lowest mean of them all.


% Profiling of clusters: Use class variable as a response variable to analyze
% conditional distributions of variables to clusters and eventual statistical
% tests to assess which variables are significant in each cluster. Detect
% commonalities of each cluster and differences between clusters. What is
% intrinsic of each cluster? What distinguishes clusters among them?

% Profiling graphs, CPGs, multiple boxplots, bivariate barplots, descriptive by
% groups, etc...

% For selected relevant variables, you can also add specific profiling tests to
% complete clusters interpretation

% Synthesize the result of the classes’ interpretation process into a set of
% templates characterizing the clusters, one template per cluster
