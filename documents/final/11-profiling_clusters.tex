% vim: spell spelllang=en:
%! TEX root = **/00-main.tex

\section{Profiling of clusters}%
\label{sec:profiling_of_clusters}

\newcommand{\profiling}[3]{
\begin{figure}[H]
    \centering
    \includegraphics[width=0.7\textwidth]{prof-c3-#1-#2}
    \caption{#3}%
    \label{fig:#1-#2}
\end{figure}
}

\profiling{room_type}{percent}{room type bar}
\profiling{room_type}{side}{room type bar}
\profiling{room_type}{stack}{room type bar}

% Profiling of clusters: Use class variable as a response variable to analyze
% conditional distributions of variables to clusters and eventual statistical
% tests to assess which variables are significant in each cluster. Detect
% commonalities of each cluster and differences between clusters. What is
% intrinsic of each cluster? What distinguishes clusters among them?

% Profiling graphs, CPGs, multiple boxplots, bivariate barplots, descriptive by
% groups, etc...

% For selected relevant variables, you can also add specific profiling tests to
% complete clusters interpretation

% Synthesize the result of the classes’ interpretation process into a set of
% templates characterizing the clusters, one template per cluster
