% vim: spell spelllang=en:
%! TEX root = **/00-main.tex

\subsection{Final task assignment}%
\label{sub:division_of_tasks}

\newcommand*\rot{\rotatebox{90}}
\newcommand*\X{\ding{56}}
\newcommand*\x{{\color{gray}\ding{55}}}
\begin{table}[H]
\centering
\begin{tabular}{@{}l|c|c|c|c@{}}
             & \rot{Aleix Boné} & \rot{Eduard Bosch} & \rot{David Gili} & \rot{Albert Mercadé} \\
\toprule
\textbf{Coordination}                           &    &    &\X  &    \\ \midrule
\textbf{Motivation and Description}             & \x &    &    & \X \\ \midrule
\textbf{Data source presentation}               &    &    &    &    \\
Description of data source                      &    & \x &    & \X \\ \midrule
\textbf{Formal description of data structure}   &    &    &    &    \\
Metadata Table                                  &\x  &\x  & \X &    \\
Scope of study                                  &\X  &\x  &    &    \\ \midrule
\textbf{Description of prepossessing}           &    &\X  &    & \x \\ \midrule
\textbf{Basic statistical descriptive analysis} &    &    &    &    \\
Univariate analysis                             &    &\x  &    & \X \\
Bivariate analysis                              &    & \x &\X  &    \\
Conclusion describing data                      &\X  & \x &    &    \\ \midrule
\textbf{PCA analysis for numerical variables}   &    &    &    &    \\
Scree plot                                      &    &    &\x  & \X \\
Factorial map visualization                     &    & \X & \x &    \\ \midrule
\textbf{Hierarchical Clustering}                &    &    &    &    \\
Clustering script                               &    & \X &    & \x \\
Description of data used                        &\x  &    & \X &    \\
Clustering method and metrics used              & \X &\x  &    &    \\
Resulting Dendogram                             &    &    &\X  &\x  \\
Discussion about number of clusters             &\x  &\X  &    &    \\
Table with description of cluster size          &\x  &    &\X  &    \\ \midrule
\textbf{Profiling of clusters}                  &    &    &    &    \\
Profiling graphs                                &    &    &\X  &\x  \\
\end{tabular}
\end{table}
