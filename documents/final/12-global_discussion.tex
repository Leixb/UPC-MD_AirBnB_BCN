% vim: spell spelllang=en:
%! TEX root = **/00-main.tex

% Global discussion and general conclusions of the whole work. Analyze
% coincidences and divergences between ACP, AMC, Clustering

\section{Conclusions}%
\label{sec:conclusions}

One of the aims of the projects was to explain what variables have an effect on a customer leaving a good or bad review score. After doing the univariate analysis we found out that reviews were pretty polarized.  In fact, most of the scores were pretty high, but there was a small percentage of really bad ones. When analyzing the clusters created by HCPC, it turned out that we accumulated the really bad reviewed listings in cluster 3, while the other three contained mostly excellent scores. This made it hard for us to determine the characteristics of a good review. Because of that, we thought that it would be appropriate to study the characteristics of the listings that had bad review scores, since they were more distinguished and less broad. We concluded that most bad reviews tend to be posted on listings not owned by a superhost, that usually has a small number of preview reviews. We also found out that listings for shared rooms where more likely to end up having a bad review.

We also managed to conclude that cluster 4 contained mostly bigger and more expensive listings that could accommodate more guests. This is interesting but it doesn't help us answer any of the questions we had proposed at the beginning of the project.
In the opposite side of the spectrum,  cluster number 1 contains mostly the listings for the smaller apartments and private rooms, with good reviews and cheaper prices. This kind of listing seems to be the most popular one among Airbnb in Barcelona, which in our experience completely makes sense.

We have analysed further clusters 1,2 and 4 as to find any major difference between them. Although some anecdotal facts have been found we don't think there's enough evidence to derive any further substantial differences.