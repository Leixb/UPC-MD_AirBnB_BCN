% vim: spell spelllang=en:
%! TEX root = **/00-main.tex

% Detailed description of Preprocessing and data preparation. Please be sure to
% justify all decisions made.

\section{Preprocessing}%
\label{sec:preprocessing}

% TODO : aixo es copiat tal qual del D3. S'ha de canviar perk sigui el que
%        realment vam acabar fent.

\subsection{Steps and Decisions}%
\label{sub:steps-decisions}

\subsubsection{Formatting issues, building software context}

We haven't encountered any formatting issues or problems with R not recognising
any rows, columns or variable types.

\subsubsection{Determining working matrix}%
\label{ssub:work_matrix}

We discarded many variables from the original data set that didn't provide any
useful information to our project such as URLs and ids or that couldn't be
categorised or were uninteresting such as the \texttt{host\_name}, name and
description of the listing, etc. Furthermore, some columns were duplicated with
different variable names, such as \texttt{host\_listings\_count} and
\texttt{host\_total\_listings\_count}, we removed those redundancies as well.

\subsubsection{Identification and treatment of missing data}

The dataset provided had already all missing data encoded as \NA's. Therefore no
more missing were found encoded as other values. When looking at the missing
values, they appear to be random. Some effort was put into analysing if the missing values had a correlation with the date of the entry, for example if older entries contained more missing data. However we couldn't find any pattern that supported those claims.

Most variables have a missing percentage between 0-5\%. However there are also
variables like \texttt{host\_response\_rate\_cat} with much higher percentages,
reaching up to 35\%. We studied each variable accordingly and performed different missing data treatment techniques.

Categorical with a large number of missing data were analysed and we found out that the best 

Numerical variables were treated using the Knn algorithm. This decision was made because our dataset already contained many numerical variables with no NA's. Therefore Knn seemed the best suited algorithm.

\subsubsection{Identification and treatment of outliers}

When looking at the boxplots of most of the numerical variables in the dataset
we can observe that there are a few outliers. Some of them, although differing
from the general distribution of the values are to be expected. For example the
outliers we find in price, beds or score could easily be explained (maybe they
correlate to the \texttt{room\_type} variable)

On the other hand we also found some outliers that correspond to extreme values
that are not expected. For example in \texttt{maximum\_nights\_avg\_ntm} we
found some rows with the value 2147483647, which corresponds to the maximum
int32 allowed. If we convert that value into years we would get a result that is
unreasonable. After doing some research we found out that the maximum number of nights is limited by AirBnb to 1125. We decided to consider all entries that surpassed that number as errors. 

We have decided to allow for some outliers as long as they are few and
justifiable, in these cases we will try to correlate them to other variables.

\subsubsection{Identification and treatment of errors}

After inspecting the data we have not found errors apart from the one abnormally
large values mentioned before. In those cases we decided to change those values to NA's and then impute those values with the knn algorithm.

\subsubsection{Feature selection/extraction, dimensional reduction}

The \airbnb dataset contained a large amount of variables. First of all we
decided which of them would made it in our working matrix following the steps
detailed in \ref{ssub:work_matrix}. Upon further investigation we decided to
discard additional variables like \texttt{host\_has\_profile\_pic} because the
information it provided wasn't relevant to this project. Of those variables that
made it into our working matrix there are several more relevant than others.

\begin{comment} % TODO posar aixo?

\subsubsection{Instance selection}

\subsubsection{Data transformation}

\end{comment}

\subsubsection{Derivation of new variables}

We derived some new variables from the data set in order to flourish new
categorical variables.

First, starting with the \texttt{host\_since} variable, which gave us the date
the host signed up to \airbnb to list his property, we extracted two new
variables: \texttt{host\_since\_year} and \texttt{host\_since\_season}, both
categorical. \texttt{host\_since\_year} categorizes the host sign up dates into
their particular year and since. The variable \texttt{host\_since\_season}
represents the season of the year during which the host signed up. For the
latter, the modalities are Winter, Spring, Summer and Autumn.

We have also categorized \texttt{host\_response\_rate} and
\texttt{host\_acceptance\_rate} as we converted them from a numerical variables
measured in percentages to categorical ones with the following modalities: very
low, low, average, high and very high.

These were all the variable derivations we carried out in our data set for this
project. They were all motivated by the necessity to have more categorical
variables in our data set, as it had plenty numerical and binary variables but
was short of categorical ones.

% Enumerate which steps of the preprocessing process are used with your
% particular data (consider the steps proposed in slide 4 of Preprocessing
% Slides in Theme 2. Data Preparation. Remember that you have the whole complete
% information on preprocessing in the reference text from the linkSurvey of
% preprocessing. Reference paper (MIE 2001) from complementary materials section
% of Theme 2. Data Preparation)

% List and justify all decisions taken for each preprocessing step

% Additional descriptive statistics of variables that have been modified or
% created by preprocessing
