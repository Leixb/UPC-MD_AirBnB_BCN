% vim: spell spelllang=en:
%! TEX root = **/00-main.tex

% Basic statistical descriptive analysis

\section{Basic statistical descriptive analysis}%
\label{sec:basic_statistical_descriptive_analysis}

% Univariate for all the variables included in the study (half a page per variable)
\subsection{Univariate analysis}%
\label{sub:univariate_analysis}


\newcommand{\fign}[2]{
    \fig{#1-hi_bp}{#2}
    \input{../../analysis/tables/#1-ext_sum}
}

\newcommand{\figf}[2]{
\begin{figure}[H]
    \begin{minipage}{0.5\linewidth}
            \centering
            \includegraphics[width=\linewidth]{#1-bar}
            \caption{Barplot chart of #2}%
            \label{fig:#1-bar}
    \end{minipage}
    \begin{minipage}{0.5\linewidth}
            \centering
            \includegraphics[width=\linewidth]{#1-pie}
            \caption{Pie chart of #2}%
            \label{fig:#1-pie}
    \end{minipage}
\end{figure}
    \input{../../analysis/tables/#1-freq}
}

\begin{comment}

\fign{accommodates}{Histogram + Boxplot of Accommodates}

In the boxplot we can that there are some outliers. However the distribution of the variable seems to be in line with socioeconomic distributions. The majority of listing have low to medium capacities (small-medium houses), while there's only a fraction of high capacities (big houses). 


\fign{bedrooms}{Histogram + Boxplot of bedrooms}

The majority of listing contains between 1-4 bedrooms, being the mean around 1.6. There are
some outliers, but they can somewhat be explained due to the same reasons mentioned in the accommodates analysis.


\fign{beds}{Histogram + Boxplot of beds}

The distribution of beds and the outliers seem to follow a similar distribution as bedrooms.
This is to be expected as those to variables should be correlated. 

\end{comment}
%\fign{availability_30}{Histogram + Boxplot of availability 30}
%\fign{availability_60}{Histogram + Boxplot of availability 60}
%\fign{availability_90}{Histogram + Boxplot of availability 90}
%\fign{availability_365}{Histogram + Boxplot of availability 365}

\figf{host_acceptance_rate_cat}{host acceptance rate cat}
%\figf{host_acceptance_rate_cat}{host acceptance rate cat}
%\figf{host_has_profile_pic}{host has profile pic}
%\figf{host_identity_verified}{host identity verified}
%\figf{host_is_superhost}{host is superhost}
%\fign{host_listings_count}{Histogram + Boxplot of host listings count}
%\figf{host_response_rate_cat}{host response rate cat}
%\figf{host_response_time}{host response time}
%\figf{host_since_season}{host since season}
%\figf{host_since_year}{host since year}
%\figf{instant_bookable}{instant bookable}
%\fign{maximum_nights_avg_ntm}{Histogram + Boxplot of maximum nights avg ntm}
\fign{minimum_nights_avg_ntm}{Histogram + Boxplot of minimum nights avg ntm}
%\figf{neighbourhood_group_cleansed}{neighbourhood group cleansed}
%\fign{number_of_reviews}{Histogram + Boxplot of number of reviews}
%\fign{number_of_reviews_l30d}{Histogram + Boxplot of number of reviews l30d}
%\fign{number_of_reviews_ltm}{Histogram + Boxplot of number of reviews ltm}
\fign{price}{Histogram + Boxplot of price}
\fig{price-hi_bp-tallat500}{Histogram + Boxplot of price}


\fign{review_scores_accuracy}{Histogram + Boxplot of review scores accuracy}
\fign{review_scores_cleanliness}{Histogram + Boxplot of review scores cleanliness}
\fign{review_scores_location}{Histogram + Boxplot of review scores location}
\fign{review_scores_rating}{Histogram + Boxplot of review scores rating}

We can see that the majority of reviews are considered positive. The mean in fact is 
a 91.07. That would explain why a common technique among renters is to consider 
any review lower than 90 as negative. 

We can see as well that there is some polarization, being the most common rating among 
negative reviews a 0 and the most common among positive a 10. We suppose polarization occurs
because users are more likely to leave reviews when they had either really good or 
really bad experiences.


\fign{review_scores_value}{Histogram + Boxplot of review scores value}
\fign{reviews_per_month}{Histogram + Boxplot of reviews per month}


\pagebreak
\subsubsection{Room Type}

\figf{room_type}{room type}

The vast majority of listings correspond to entire homes/apartments or private rooms.

\pagebreak
% Bivariate when relevant (half a page per pair of variables)
\subsection{Bivariate analysis}%
\label{sub:bivariate_analysis}

% When required, please include descriptives before and after preprocessing

% Conclude the section with one paragraph describing how is your data
