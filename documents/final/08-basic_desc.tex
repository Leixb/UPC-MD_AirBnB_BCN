% vim: spell spelllang=en:
%! TEX root = **/00-main.tex

% Basic statistical descriptive analysis

\section{Basic statistical descriptive analysis}%
\label{sec:basic_statistical_descriptive_analysis}

% Univariate for all the variables included in the study (half a page per variable)
\subsection{Univariate analysis}%
\label{sub:univariate_analysis}


\newcommand{\hibp}[2]{
    \begin{figure}[H]
        \centering
        \includegraphics[width=0.8\linewidth]{#1-hi_bp}
        \caption{Histogram \& Boxplot of #2}%
        \label{fig:#1}
    \end{figure}
}

\newcommand{\sssfig}[1]{
    \pagebreak
    \subsubsection{#1}%
    \label{ssub:#1}
}

\newcommand{\fign}[2]{
\sssfig{#2}
    \hibp{#1}{#2}
    \begin{table}[H]
        \centering
        \caption{Extended summary statistics of #2}%
        \label{tab:#2}
        \input{../../analysis/tables/#1-ext_sum}
    \end{table}
}

\newcommand{\figf}[2]{
\sssfig{#2}
\begin{figure}[H]
    \begin{minipage}{0.5\linewidth}
            \centering
            \includegraphics[width=\linewidth]{#1-bar}
            \caption{Barplot chart of #2}%
            \label{fig:#1-bar}
    \end{minipage}
    \begin{minipage}{0.5\linewidth}
            \centering
            \includegraphics[width=\linewidth]{#1-pie}
            \caption{Pie chart of #2}%
            \label{fig:#1-pie}
    \end{minipage}
\end{figure}
\begin{table}[H]
    \centering
    \caption{Table of #2 frequency}%
    \label{tab:#1}
    \input{../../analysis/tables/#1-freq}
\end{table}
}

\begin{comment}

\fign{accommodates}{Accommodates}

In the boxplot we can that there are some outliers. However the distribution of the variable seems to be in line with socioeconomic distributions. The majority of listing have low to medium capacities (small-medium houses), while there's only a fraction of high capacities (big houses). 


\fign{bedrooms}{bedrooms}

The majority of listing contains between 1-4 bedrooms, being the mean around 1.6. There are
some outliers, but they can somewhat be explained due to the same reasons mentioned in the accommodates analysis.


\fign{beds}{beds}

The distribution of beds and the outliers seem to follow a similar distribution as bedrooms.
This is to be expected as those to variables should be correlated. 


\fign{availability_30}{availability 30}

If we look at the availability in the next 30 days after the data was taken, 
we can clearly see some polarization. There are some rentals with 0 available days and some close to all days. This polarization explains the high variance
of this variable.


%\fign{availability_60}{availability 60}
%\fign{availability_90}{availability 90}
\fign{availability_365}{availability 365}

The availability of the year follows a similar distribution to the monthly one.
The only difference are some peaks at 90 days and half a year, which both are 
likely numbers for a host to rent it's listing.


\figf{host_acceptance_rate_cat}{host acceptance rate cat}

Up to 66\% of the hosts have a very high chance of accepting a customer. There is a large number of NA's as well.


\figf{host_has_profile_pic}{host has profile pic}

As seen in figure \ref{fig:host_has_profile_pic-pie}, over 99\% of host with listings in 
Barcelona have a profile picture, in fact only 55 out of 20,476 hosts don't
have a picture. Therefore, we don't expect this variable to provide significant
insight into answering the question we are posing.

\figf{host_identity_verified}{host identity verified}

\Cref{fig:host_identity_verified-pie} shows that only 74\% of hosts have
had their identity verified by \airbnb, meaning that the other 25\% isn't.
This can be expected since the verification process can be complicated in some
cases, even requiring the host to provide an official document or additional
pictures of himself.

\figf{host_is_superhost}{host is superhost}

As seen in figure \ref{fig:host_is_superhost-pie}, only the 19\% of hosts are superhosts. 
This may be the case because the conditions to achieve this status are rather hard.

\fign{host_listings_count}{host listings count}


Figure \ref{fig:host_listings_count} shows that the majority of hosts only have 
a small amount of listings, being the median 2. However we can see that there is
a lot of variability with many outliers. These outliers might correspond to 
businesses that own lots of listings.

% TODO
%\fig{host_listings_count-hi_bp-tallat100}{host listings count}


\figf{host_response_rate_cat}{host response rate cat}

This variable conveys how attentive the host is, as it tells us to what degree
the host replies to messages received inquiring about its listings. As seen
in figure \ref{fig:host_response_rate_cat-pie}, 55\% of hosts have a very high
response rate, that is they reply to more than 80\% of messages they receive.
It should be noted that this variable has a high share of missing data, 
approximately 34\%.


\figf{host_response_time}{host response time}

Figure \ref{fig:host_response_time-bar} shows that the quicker the response, the
more number of entries it has. A fact to point out as well is the large number
of NA's.


\figf{host_since_season}{host since season}

To our surprise there number of new hosts listings remain more or less equal over
the four seasons. In figure \ref{fig:host_since_season-bar} we can see that Spring/Summer
still have a bigger influx but not as much as we had initially hypothesised.


\figf{host_since_year}{host since year}

In figure \ref{fig:host_since_year-pie} we can see that except for the first years the among of new hosts has more or less remained constant. However 
in 2019 there has been another spike of new hosts.


\figf{instant_bookable}{instant bookable}

As seen in figure \ref{fig:instant_bookable-bar} there are nearly as many instant bookable listings as not.

\end{comment}

%\fign{maximum_nights_avg_ntm}{maximum nights avg ntm}

%\fig{maximum_nights_avg_ntm-hi_bp-post}{maximum nights avg ntm post}

%\fign{minimum_nights_avg_ntm}{minimum nights avg ntm}

\figf{neighbourhood_group_cleansed}{neighbourhood group cleansed}

Looking at \ref{fig:neighbourhood_group_cleansed-pie} we can see that L'Eixample
and Ciutat Vella are the most popular neighborhoods for listing.

\fign{price}{price}
\fig{price-hi_bp-tallat500}{price}

In the figure \ref{fig:price-hi_bp-tallat500} we can see that there are some extreme outliers. To visualize the distribution better we have cut the histogram at price 500. Having done that we can see that the price distribution seems to follow a chi-square. We were surprised to find out that the mean price is 
around \$86.34.


\fign{review_scores_rating}{review scores rating}

We can see that the majority of reviews are considered positive. The mean in fact is 
a 91.07. That would explain why a common technique among renters is to consider 
any review lower than 90 as negative. 

We can see as well that there is some polarization, being the most common rating among 
negative reviews a 0 and the most common among positive a 10. We suppose polarization occurs
because users are more likely to leave reviews when they had either really good or 
really bad experiences.


\fign{review_scores_value}{review scores value}

This variable follows a similar distribution as the other reviews seen so far. Despite that
it has the peculiarity of having less perfect scores.


\fign{review_scores_location}{review scores location}

The location score seems to be the one with most positive reviews. In fact the mean is 9.61, almost 0.5 higher than the overall rating. We found as well
very few negative reviews, being in fact the median and Q3 both a 10.


\fign{review_scores_accuracy}{review scores accuracy}

The distribution of this variable is really similar to the review score ratings.


\fign{review_scores_cleanliness}{review scores cleanliness}

The distribution of this variable is really similar to the review score ratings.


\fign{number_of_reviews}{number of reviews}
\fign{number_of_reviews_l30d}{number of reviews l30d}
\fign{number_of_reviews_ltm}{number of reviews ltm}

\fign{reviews_per_month}{reviews per month}

When looking at \ref{fig:reviews_per_month} we can clearly see a decreasing curve with its mean around 1.17. We found some outliers with the max number of reviews per month being 21. Despite it being quite larger than the mean we think it is still plausible to have 21 reviews over a 30 day span.

\figf{room_type}{room type}

The vast majority of listings correspond to entire homes/apartments or private rooms.

\pagebreak
% Bivariate when relevant (half a page per pair of variables)
\subsection{Bivariate analysis}%
\subsubsection{Host since year vs host listings count}

\fig{bivar-host_since_year-host_listings_count}{Number of listings depending on the year the host joined Airbnb}

Looking at \cref{fig:bivar-host_since_year-host_listings_count} we can see that there is no clear correlation between the number of listings a host has and the time they have been in the platform.It seems like most of the users have a small amount of listings. 
But the small group that joined the platform in the years 2009, 2010 and 2011 seems to include some owners with upwards of 100 and 150 listings. This seems to confirm that most of the early adopters of the platform were businesses or big owners that gathered a lot of buildings. We can also see that some of the hosts with more listings joined in the recent years, probably joining the platform after seeing great results from others' operations. 


\pagebreak
\subsubsection{Host since year vs price per night}

\fig{bivar-host_since_year-price}{Price per night depending on the year the host joined \airbnb}

Looking at \cref{fig:bivar-host_since_year-price} we can see that again, there is no relation between the price of the listings and the year the host joined, there are a few differences on price, all in the neighbourhood of 80-125 dollars a night, but we also see the same tendency we saw in the last graph, the median price for the listings posted by hosts who joined during years 2009, 2010 and 2011 seems to be a bit higher than the median price for the hosts who joined in subsequent years.
\pagebreak

\subsubsection{Host since year vs room type}

\fig{bivar-host_since_year-room_type}{Room type depending on the host's joining year}

What we wanted to find out with \cref{fig:bivar-host_since_year-room_type} was if there was any initial promotion within a certain group of hosts, something along the lines of contacting different hotel chains to promote the platform, which at that time was still small, and give them certain advantages to use \airbnb.
We would have expected, if that was the case, a significantly bigger share of the early years' hosts to be listing hotel rooms instead of private rooms or entire apartments.
This obviously was not the case, which is interesting, since it indicates that \airbnb did not grow through the kind of techniques we had expected.


\pagebreak
\subsubsection{Number of reviews vs review score rating}

\fig{bivar-number_of_reviews-review_scores_rating}{Score rating vs number of reviews}

 In this case, \cref{fig:bivar-host_since_year-room_type} shows how the number of reviews affects the total rating. We can see that when there are not many reviews those can be anywhere within range, but once the listings get a significant amount of reviews it is harder and harder to find listings with a low rating. This, we believe, is due to people's tendency to give polarized scores and avoid booking the listings with really low scores. 

\pagebreak
\subsubsection{Reviews per month vs review score rating vs price}

\fig{bivar-reviews_per_month-review_scores_rating}{Score rating vs reviews per month}

 To see if there is a difference, we have created  \cref{fig:bivar-reviews_per_month-review_scores_rating}, which shows how the number of reviews per month correlates to the score of the reviews a listing has, but also integrating the price of such listing. It is interesting to see that some of the most reviewed listings are not very expensive and that the tendency that we saw in \cref{fig:bivar-host_since_year-room_type} is still there, proving that the listings that get reviewed the most tend to have pretty good ratings.

\pagebreak

\fig{bivar-neighbourhood_group_cleansed-price}{Price of listings depending on the neighbourhood \airbnb}
As Barcelona dwellers we know that there are some neighborhoods more expensive to others. We want to find out
if this still holds when talking about \airbnb rents.
We can see in figure \cref{fig:bivar-neighbourhood_group_cleansed-price} that for example L'Eixample has a mean price of ...,
while Nou Barris has a mean price of .... We can see as well
that the high price outliers seem to vary between neighborhoods. This leads us to believe that although not much there is some correlation between price and neighborhood.

\pagebreak

\fig{bivar-room_type-minimum_nights_avg_ntm}{Minimum number of nights depending on room type} 
From figure \cref{fig:bivar-room_type-minimum_nights_avg_ntm} we can clearly see that hotel rooms and
shared rooms have a minimum average of nights much lower than the entire homes or private rooms.
We can see that entire homes/apartments have less variance, being the majority of values concentrated
near the mean, while in the private rooms we can see a much larger interquartile range.

\pagebreak

\label{sub:bivariate_analysis}

% When required, please include descriptives before and after preprocessing

% Conclude the section with one paragraph describing how is your data
