% vim: spell spelllang=en:
%! TEX root = **/00-main.tex
\documentclass[12pt, oneside]{article}
%\documentclass[draft, 12pt, oneside]{article}
%\documentclass[final, 12pt, oneside]{article}
\usepackage[a4paper, left=2.5cm, right=2.5cm, top=2.5cm, bottom=2.5cm]{geometry}
%\usepackage[showframe, a4paper, left=2.5cm, right=2.5cm, top=2.5cm, bottom=2.5cm]{geometry}

\usepackage[T1]{fontenc}

%% Not needed with lualatex
%\usepackage[utf8]{inputenc} % Use unicode for input
%\usepackage{lmodern}

% With lualatex:
\usepackage{fontspec}
\setmonofont[Scale=MatchLowercase]{DejaVu Sans Mono}

\usepackage{microtype}

\usepackage{polyglossia}
\setdefaultlanguage{english}

\usepackage{hyphenat}

%% Bibliography:
\usepackage[
    backend=biber,
    style=numeric,
]{biblatex}
\DeclareNameAlias{default}{last-first}

%\DeclareQuoteAlias{spanish}{catalan}

\addbibresource{biblio.bib}
%% see:
%% https://www.sharelatex.com/learn/Bibliography_management_in_LaTeX#The_bibliography_file

% For cross references
\usepackage{color, xcolor}
\usepackage[breaklinks, colorlinks = true]{hyperref}
%hyperref configuration so that it doesn't contrast so much colorlinks,
\hypersetup{
   linkcolor={black},
   citecolor={black},
   urlcolor={blue!80!black}
}

\usepackage{mathtools}                    % amsmath + more
\usepackage{amsthm}                       % Theorem enviroment
\usepackage{amssymb}                      % More symbols
\usepackage{amstext}                      % Text inside mathenv

\usepackage{relsize}                      % Bigger math with mathlarger{___}
\usepackage{nicefrac}                     % nice fractions in one line

\usepackage[export]{adjustbox}            % Adjust table size
\usepackage{float}                        % Force tables and images position (H and H!)
\usepackage{wrapfig}                      % Wrap images like in HTML

%\usepackage{tabularx, colortbl, booktabs} % Better tables
%\usepackage{longtable}                    % Multiple page table
\usepackage{xltabular, colortbl, booktabs} % longtable + tabularx (has bug with booktabs: fix below)

% bug booktabs + xltabular
\makeatletter
\def\@BTrule[#1]{%
  \ifx\longtable\undefined
    \let\@BTswitch\@BTnormal
  \else\ifx\hline\LT@hline
    \nobreak
    \let\@BTswitch\@BLTrule
  \else
     \let\@BTswitch\@BTnormal
  \fi\fi
  \global\@thisrulewidth=#1\relax
  \ifnum\@thisruleclass=\tw@\vskip\@aboverulesep\else
  \ifnum\@lastruleclass=\z@\vskip\@aboverulesep\else
  \ifnum\@lastruleclass=\@ne\vskip\doublerulesep\fi\fi\fi
  \@BTswitch}
\makeatother

% Split cell in lines and more formating options inside table
\usepackage{array, multirow, multicol, makecell}

%\usepackage{subcaption}                     % Subfigures
%\usepackage[framemethod=tikz]{mdframed}     % Custom frames

\usepackage[bottom]{footmisc} % Footnotes at bottom of page

%\usepackage[alsoload=hep]{siunitx}          % SI units and uncertainties
%\sisetup{locale = FR}                       % Commas and so on for spanish
%\sisetup{separate-uncertainty=true}
%\sisetup{
  %per-mode=fraction,
  %fraction-function=\nicefrac
%}

%\usepackage{tikz}
%%\usetikzlibrary{arrows}
%%\usetikzlibrary{scopes}
%\usetikzlibrary{babel}

%% Custom Math operators (functions not in italic in math mode):
%\DeclareMathOperator{\cis}{cis}

% Set pages to landscape
\usepackage{pdflscape}
\usepackage{afterpage}

\usepackage{pdfpages}

\usepackage{comment}        % multiline comments

\usepackage{pifont}         % some fancy symbols

\usepackage{xspace}         % to create commands with space at end

%\usepackage{listings}       % For code listings (prefer minted)
\usepackage{minted}         % -shell-escape
\definecolor{codeBg}{HTML}{FFFDE7}
\setminted[r]{
    %style=pastie,
    frame=lines,
    framesep=3mm,
    linenos,
    breaklines=true,
    encoding=utf8,
    fontsize=\footnotesize,
    bgcolor=codeBg
}

\usepackage{pgfgantt}           % Gantt chart

\newcommand{\whitepage}{        % white page without adding to pagenumber
    \clearpage\thispagestyle{empty}\addtocounter{page}{-1} \newpage \clearpage
}
\newcommand{\ts}{\textsuperscript}

\usepackage{tocbibind}

\usepackage[justification=centering]{caption}

\usepackage[english]{cleveref}
\usepackage{csquotes}       % For bibliography quotations

\usepackage{subcaption}

\title{MD - \airbnb Barcelona Listings}
\author{
    Aleix Bon\'e\\
    Eduard Bosch\\
    David Gili\\
    Albert Mercad\'e
}
\date{\today}

\setbeamercolor{background canvas}{bg=white}

\graphicspath{{../../analysis/plots/}{images}}

\newcommand{\airbnb}{\emph{Airbnb}\xspace}
\newcommand{\NA}{\emph{NA}\xspace}

\newcommand{\profiling}[3]{
\begin{figure}[H]
    \centering
    \includegraphics[width=0.8\textwidth]{prof-c3-#1-#2}
    \caption{#3}%
    \label{fig:prof-#1-#2}
\end{figure}
}

\newcommand{\fig}[3][0.6]{
    \begin{figure}[H]
        \centering
        \includegraphics[width=#1\linewidth]{#2}
        \caption{#3}%
        \label{fig:#2}
    \end{figure}
}

\newcommand{\hibp}[2]{
    \begin{figure}[H]
        \centering
        \includegraphics[width=0.6\linewidth]{desc-#1-hi_bp}
        \caption{Histogram \& Boxplot of #2}%
        \label{fig:#1}
    \end{figure}
}

\newcommand{\tabn}[2]{
    \begin{table}[H]
        \centering
        \caption{Extended summary statistics of #2}%
        \label{tab:#1}
        \input{../../analysis/tables/desc-#1-ext_sum}
    \end{table}
}

\newcommand{\factorialmap}[2]{
    \begin{figure}[H]
        \centering
        \includegraphics[width=0.7\linewidth]{pca_fact-plane_#1_#2-bi}
        \caption{PCA plane #1 vs #2}%
        \label{fig:plane_#1-#2}
    \end{figure}
}

\newcommand{\fign}[2]{
    %\sssfig{#2}
    \hibp{#1}{#2}
    %\tabn{#1}{#2}
}

\newcommand{\figf}[2]{
%\sssfig{#2}
\begin{figure}[H]
    \begin{minipage}{0.39\linewidth}
            \centering
            \includegraphics[width=\linewidth]{desc-#1-bar}
    \end{minipage}
    \begin{minipage}{0.39\linewidth}
            \centering
            \includegraphics[width=\linewidth]{desc-#1-pie}
    \end{minipage}
    \caption{Barplot \& Pie chart of #2}%
    \label{fig:#1}
\end{figure}


%\begin{table}[H]
%    \centering
%    \caption{Table of #2 frequency}%
%    \label{tab:#1}
%    \input{../../analysis/tables/desc-#1-freq}
%\end{table}
}

%\hfuzz=3pt
%\vfuzz=3pt

%1. Slide with title, name of all group components, delivery date
%2. Slide with outline of talk
%3. Slide with topics addressed, goals of the work and urls from data sources, overview of BD structure and variables analyzed
%4. Slide with the Data mining process schema
%5. Slide with the descriptive analysis of one numerical variable and one qualitative variable
%6. Slide synthesizing univariate descriptive analysis
%7. Slide with additional descriptive analysis issues when relevant
%8. Slide describing preprocessing steps (if required add additional slides for any specific aspect to be commented)
%9. Slide with PCA specificacions, screeplot
%10. Slide wtith first factorial plane for PCA (eventually additional slides for othe planes retained). Lack of visibility of map
%penalizes.
%11. Slide with conlusions of PCA
%12. Slide describing the clustering process followed and resulting dendrogramm
%13. Slide describing which tools of class interpretation you have been used
%14. Slide with CPG or eventual profiling graphs or numerical information about clusters to be highlighted (whenever possible,
%synthesize important graphics in a single slide... evenctually you can add some extra slide)
%15. Slide with final class profiling (synthesis with description of class characteristics)
%16. Slide with comparison of conclusions between PCA and clustering
%17. Slide with conclusions
%18. Slide with original and final scheduling

\begin{document}

\maketitle

\section{Outline}
\begin{frame}{Outline}
\begin{scriptsize}
    \tableofcontents
\end{scriptsize}
\end{frame}

%3. Slide with topics addressed, goals of the work and urls from data sources, overview of BD structure and variables analyzed
\section{Data source}
\begin{frame}{Our Dataset}
\begin{center}
    Inside Airbnb Barcelona Listings - 24/08/2020\footnote[frame]{Dataset URL: \url{http://data.insideairbnb.com/spain/catalonia/barcelona/2020-08-24/data/listings.csv.gz}}

\vspace{5 mm}

\begin{columns}[t]
    \column{0.3\textwidth}
    20,703 data records
    
    9.01\% of missing data
    
    74 variables
    \small
    \begin{itemize}[topsep=0pt]
        \itemsep-0.25em
    	\item[--] 30 categorical
    	\item[--] 38 numerical
    	\item[--] 6 boolean
    \end{itemize}
    \normalsize
    
    \column{0.3\textwidth}
    Host information
    \small
    \begin{itemize}[topsep=0pt]
        \itemsep-0.5em
    	\item[--] Acceptance rate
    	\item[--] Listings count
    	\item[--] ...
    \end{itemize}
    \normalsize
    Listing information
    \small
    \begin{itemize}[topsep=0pt]
        \itemsep-0.5em
    	\item[--] Price
    	\item[--] Reviews
    	\item[--] ...
    \end{itemize}
    \normalsize
\end{columns}

\end{center}
\end{frame}


\begin{frame}{Goals}
\begin{quote}
The focus of the project will be to analyse whether or not factors
such as price, availability or neighbourhood impact whether or not the final
user review is positive or negative.
\end{quote}
\end{frame}

%4. Slide with the Data mining process schema
\section{Data mining process scheme}
\begin{frame}{Data mining process}
\begin{figure}[H]
    \centering
    \includegraphics[width=0.75\linewidth]{../final/images/workflow}
    %\caption*{Our data mining workflow}
\end{figure}
\end{frame}

%5. Slide with the descriptive analysis of one numerical variable and one qualitative variable
\section{Descriptive analysis}
\begin{frame}{Numerical variable}
\vspace{0.7em}
\fign{review_scores_rating}{review scores rating}
\end{frame}

\begin{frame}{Qualitative variable}
\vspace{1.5em}
\figf{neighbourhood_group_cleansed}{neighbourhood group cleansed}
\end{frame}


%6. Slide synthesizing univariate descriptive analysis
\section{Univariate descriptive analysis}
\begin{frame}{Univariate descriptive analysis}
\end{frame}

\section{Bivariate descriptive analysis}
\begin{frame}{Bivariate descriptive analysis}
\fig{bivar-reviews_per_month-review_scores_rating}{Score rating vs reviews per month}
\end{frame}

%7. Slide with additional descriptive analysis issues when relevant
\begin{frame}{Issues}
\fign{host_listings_count}{Host listings count}
\end{frame}

%8. Slide describing preprocessing steps (if required add additional slides for any specific aspect to be commented)
\section{Preprocessing}
\begin{frame}{Preprocessing steps}
\large
\begin{itemize}
     \itemsep0.5em
     \item Determining working matrix
     \item Missing 
     \begin{itemize}
         \itemsep0.25em
         \item Numerical $\rightarrow$ Knn Algorithm
         \item Categorical $\rightarrow$ New 'Unknown' category 
     \end{itemize}
     \item Outliers
     \item Errors
     \item Derivation of new categorical variables
\end{itemize}
\normalsize
\end{frame}

%9. Slide with PCA specificacions, screeplot
\section{PCA}
\begin{frame}{Screeplot}
\begin{figure}[H]
    \centering
    \includegraphics[width=0.7\linewidth]{pca_fact-screeplot} % PCA-inertia_cum
    \caption{PCA inertia}%
    \label{fig:pca_inertia}
\end{figure}
\end{frame}

%10. Slide wtith first factorial plane for PCA (eventually additional slides for othe planes retained). Lack of visibility of map
%penalizes.
\begin{frame}{First factorial plane}
\factorialmap{1}{2}
\end{frame}

\begin{frame}{factorial plane 1 v 3} %  o lo k sigui
\factorialmap{1}{3}
\end{frame}

\begin{frame}{factorial plane 2 v 4} %  o lo k sigui
\factorialmap{2}{4}
\end{frame}

%11. Slide with conlusions of PCA
\begin{frame}{Conclusions}
Variables that contribute the most to each factorial axis:
\begin{itemize}
     \item Axis 1: accommodates, beds and bedrooms (direct relation)
     \item Axis 2: review scores ratings (direct relation)
     \item Axis 3: number of reviews (direct relation)
     \item Axis 4: availability (direct relation)
\end{itemize}
\end{frame}

%12. Slide describing the clustering process followed and resulting dendrogramm
\section{Clustering}
\begin{frame}{Process}
    \begin{figure}[H]
    \centering
    \includegraphics[width=0.7\linewidth]{factor_map}
    \caption{Clusters on PCA}%
    \label{fig:clusters-pca}
    \end{figure}
\end{frame}

\begin{frame}{Resulting Dendogram}
\begin{figure}[H]
    \centering
    %\includegraphics[width=0.8\linewidth]{dendo}
    \includegraphics[width=0.8\linewidth]{cluster-dendo-h3}
    %\caption{Cluster Dendogram}%
    \label{fig:dendogram-final}
\end{figure}
\end{frame}


%13. Slide describing which tools of class interpretation you have been used
\begin{frame}{Class interpretation Dendogram}
\end{frame}

%14. Slide with CPG or eventual profiling graphs or numerical information about clusters to be highlighted (whenever possible,
%synthesize important graphics in a single slide... evenctually you can add some extra slide)
\section{Profiling}
\begin{frame}{Profiling}

\begin{columns}
\begin{column}{0.5\textwidth}
\profiling{review_scores_rating}{vi}{Review scores rating}
\end{column}
\begin{column}{0.5\textwidth}  %%<--- here
\profiling{reviews_per_month}{bp}{Review per month}
\end{column}
\end{columns}

\end{frame}

%15. Slide with final class profiling (synthesis with description of class characteristics)
%16. Slide with comparison of conclusions between PCA and clustering


%17. Slide with conclusions
\section{Conclusions}
\begin{frame}{Conclusions}
\end{frame}

%18. Slide with original and final scheduling
\section{Working plan}
\begin{frame}{Original scheduling}
\centering
\vspace{1.4em}
\scalebox{0.25}{%\begin{minipage}{1.20\textwidth}
% vim: spell spelllang=en:
%! TEX root = **/00-main.tex

\newgeometry{top=0.5cm, bottom=0.5cm, left=0.5cm, right=0.5cm}
\begin{landscape}

\section{Working plan}%
\label{sec:working_plan}

\null
\vspace{1em}
{
\centering
\subsection{Initial Gantt}%
\label{sub:ini_gantt}
\vspace{2em}
\par}

\begin{center}
\begin{ganttchart}[
vgrid={*{6}{draw=none}, dotted},
x unit=.75cm,
y unit title=1cm,
y unit chart=1cm,
    time slot format=isodate
    ]{2020-09-15}{2020-10-07}
\gantttitlecalendar{month=name, day}
\ganttnewline

\ganttgroup{Motivation and Description}{2020-09-16}{2020-09-23}
\ganttnewline

\ganttgroup{Data source presentation}{2020-09-16}{2020-09-23}
\ganttnewline

\ganttbar{Description of data source}{2020-09-16}{2020-09-23}
\ganttnewline

\ganttgroup{Formal description of data structure}{2020-09-23}{2020-09-30}
\ganttnewline
\ganttbar{Metadata Table}{2020-09-23}{2020-09-25}
\ganttnewline
\ganttbar{Scope of study}{2020-09-25}{2020-09-30}
\ganttnewline

\ganttgroup{Description of preprocessing}{2020-09-23}{2020-09-27}
\ganttnewline

\ganttgroup{Basic statistical descriptive analysis}{2020-09-27}{2020-10-07}
\ganttnewline
\ganttbar{Univariate analysis}{2020-09-27}{2020-09-30}
\ganttnewline
\ganttbar{Bivariate analysis}{2020-10-01}{2020-10-04}
\ganttnewline
\ganttbar{Conclusion describing data}{2020-10-04}{2020-10-07}
\ganttnewline
\end{ganttchart}
\end{center}

\pagebreak
\null
\vspace{3.5em}
\begin{center}
\begin{ganttchart}[
vgrid={*{6}{draw=none}, dotted},
x unit=.85cm,
y unit title=1cm,
y unit chart=1cm,
    time slot format=isodate
    ]{2020-10-07}{2020-10-28}
\gantttitlecalendar{month=name, day}
\ganttnewline

\ganttgroup{PCA analysis for numerical variables}{2020-10-07}{2020-10-14}
\ganttnewline
\ganttbar{Scree plot}{2020-10-07}{2020-10-11}
\ganttnewline
\ganttbar{Factorial map visualization}{2020-10-11}{2020-10-14}
\ganttnewline

\ganttgroup{Hierarchical Clustering}{2020-10-14}{2020-10-21}
\ganttnewline
\ganttbar{Clustering script}{2020-10-14}{2020-10-19}
\ganttnewline
\ganttbar{Description of data selected}{2020-10-19}{2020-10-21}
\ganttnewline
\ganttbar{Clustering method and metrics used}{2020-10-19}{2020-10-21}
\ganttnewline
\ganttbar{Resulting Dendogram}{2020-10-19}{2020-10-21}
\ganttnewline
\ganttbar{Discussion about number of clusters}{2020-10-19}{2020-10-21}
\ganttnewline
\ganttbar{Table with description of cluster size}{2020-10-19}{2020-10-21}
\ganttnewline

\ganttgroup{Profiling of clusters}{2020-10-21}{2020-10-28}
\ganttnewline
\ganttbar{Profiling graphs}{2020-10-21}{2020-10-28}
\ganttnewline

\end{ganttchart}
\end{center}
\end{landscape}

\restoregeometry

%\end{minipage}
}
\end{frame}

\begin{frame}{Final scheduling}
\centering
\vspace{1.4em}
\scalebox{0.25}{
% vim: spell spelllang=en:
%! TEX root = **/00-main.tex

\newgeometry{top=0.5cm, bottom=0.5cm, left=0.5cm, right=0.5cm}
\begin{landscape}

\null
\vspace{1em}
{
\centering
\subsection{Final Gantt}%
\label{sub:fin_gantt}
\vspace{2em}
\par}

\begin{center}
\begin{ganttchart}[
vgrid={*{6}{draw=none}, dotted},
x unit=.75cm,
y unit title=1cm,
y unit chart=1cm,
    time slot format=isodate
    ]{2020-09-15}{2020-10-09}
\gantttitlecalendar{month=name, day}
\ganttnewline

\ganttgroup{Motivation and Description}{2020-09-16}{2020-09-21}
\ganttnewline

\ganttgroup{Data source presentation}{2020-09-16}{2020-09-22}
\ganttnewline

\ganttbar{Description of data source}{2020-09-16}{2020-09-22}
\ganttnewline

\ganttgroup{Formal description of data structure}{2020-09-23}{2020-09-30}
\ganttnewline
\ganttbar{Metadata Table}{2020-09-23}{2020-09-25}
\ganttnewline
\ganttbar{Scope of study}{2020-09-25}{2020-09-30}
\ganttnewline

\ganttgroup{Description of preprocessing}{2020-09-23}{2020-09-28}
\ganttnewline

\ganttgroup{Basic statistical descriptive analysis}{2020-09-27}{2020-10-09}
\ganttnewline
\ganttbar{Univariate analysis}{2020-09-27}{2020-10-02}
\ganttnewline
\ganttbar{Bivariate analysis}{2020-10-03}{2020-10-08}
\ganttnewline
\ganttbar{Conclusion describing data}{2020-10-08}{2020-10-09}
\ganttnewline
\end{ganttchart}
\end{center}

\pagebreak
\null
\vspace{3.5em}
\begin{center}
\begin{ganttchart}[
vgrid={*{6}{draw=none}, dotted},
x unit=.85cm,
y unit title=1cm,
y unit chart=1cm,
    time slot format=isodate
    ]{2020-10-07}{2020-10-28}
\gantttitlecalendar{month=name, day}
\ganttnewline

\ganttgroup{PCA analysis for numerical variables}{2020-10-07}{2020-10-12}
\ganttnewline
\ganttbar{Scree plot}{2020-10-07}{2020-10-10}
\ganttnewline
\ganttbar{Factorial map visualization}{2020-10-10}{2020-10-12}
\ganttnewline

\ganttgroup{Hierarchical Clustering}{2020-10-12}{2020-10-20}
\ganttnewline
\ganttbar{Clustering script}{2020-10-12}{2020-10-16}
\ganttnewline
\ganttbar{Description of data selected}{2020-10-17}{2020-10-18}
\ganttnewline
\ganttbar{Clustering method and metrics used}{2020-10-17}{2020-10-18}
\ganttnewline
\ganttbar{Resulting Dendogram}{2020-10-17}{2020-10-19}
\ganttnewline
\ganttbar{Discussion about number of clusters}{2020-10-19}{2020-10-20}
\ganttnewline
\ganttbar{Table with description of cluster size}{2020-10-19}{2020-10-20}
\ganttnewline

\ganttgroup{Profiling of clusters}{2020-10-20}{2020-10-26}
\ganttnewline
\ganttbar{Profiling graphs}{2020-10-20}{2020-10-26}
\ganttnewline

\end{ganttchart}
\end{center}
\end{landscape}

\restoregeometry
 
}
\end{frame}







% Video frances de HCPC
% https://www.youtube.com/watch?v=4XrgWmN9erg

%% vim: spell spelllang=en:
%! TEX root = **/00-main.tex

% Cover with title of work, name of course, data and list of working
% team members by alphabetical order of family name

\thispagestyle{empty}
\clearpage
\setcounter{page}{-1}

\begin{titlepage}
{
    \centering
    \null
    \vfill
    {\Large Data Mining\par}
    \vspace{2em}
    {\Huge \bfseries
        \airbnb Barcelona Listings
    \par}
    \vspace{2em}
    {\large \scshape
        \today
    \par}
    \vfill
\begin{center}

\end{center}
    \vspace{3cm}

    \vfill
    {\raggedleft \large
Aleix Boné\\
Eduard Bosch\\
David Gili\\
Alber Mercadé\\
        \par}
}
\end{titlepage}

%
%\pagenumbering{Roman}
%
%%\setcounter{tocdepth}{2}
%\tableofcontents
%
%\pagebreak
%\listoffigures
%
%\pagebreak
%\listoftables
%
%\clearpage
%\pagenumbering{arabic}
%
%\setlength{\parskip}{1em}

%% vim: spell spelllang=en:
%! TEX root = **/00-main.tex

%Motivation of the work and general description of the problem to be analyzed (max one page)

\section{Motivation}%
\label{sec:motivation}

Airbnb
%% vim: spell spelllang=en:
%! TEX root = **/00-main.tex

% Data Source presentation (repeating what was delivered in first part) (one paragraph)

\section{Data source}%
\label{sec:data_source}

The \airbnb data for Barcelona listings that we use in this project comes from a data
set\footnote{Barcelona listings data set:
\url{http://data.insideairbnb.com/spain/catalonia/barcelona/2020-08-24/data/listings.csv.gz}} 
created by Inside AirBnb\footnote{Inside Airbnb: \url{http://insideairbnb.com/}},
which aims to present \airbnb data for most major cities all around the globe in
order to provide valuable insight into how \airbnb is being used in each one of them. It 
extracts this data directly from \airbnb's website and the version of the data set 
we are working with is a snapshot of all Barcelona listings that were available 
on the 24th of August 2020.


\subsection{About the data}%

The data set has 20,703 entries and 74 variables, out of which 
%% vim: spell spelllang=en:
%! TEX root = **/00-main.tex


% Formal description of Data structure and metadata

\section{Description of data}%
\label{sec:description_of_data}

\subsection{Description of initial data matrix}

The data set has 20,703 entries and 74 variables, out of which 30 are categorical,
38 are numerical and 6 are boolean, meaning that there are 1,532,022 data entries.
The share of missing data is 9.01\%, or 138,007
entries in absolute terms. We should note that two variables, \texttt{bathrooms} and
\texttt{calendar\_updated}, have NA values for all records and hence are useless.
If we remove those two variables, the share of missing data drops to
around 6.5\%.
Furthermore, for a few variables such as \texttt{neighbourhood} and all variables
related to reviews we have a significant share of NA entries, around 25 and 30\%
or 5,000 to 7,000 in absolute terms. The rest of the variables have largely no
missing data and the few that do are below 3\%.

Regarding the hosts, it tells us about, among other things, when they joined
\airbnb, where they live, how long they take to respond
messages, their acceptance and response rates, whether they are superhosts, how
many listings they have or whether they have a profile picture.

The information it supplies concerning the listings includes its
neighbourhood, its coordinates, the type of property it is, the number of people
it accommodates, the number of bedrooms, beds and bathrooms, amenities, price,
the minimum and maximum nights it can be rented, its availability and reviews.

Additionally, the data set contains variables that don't provide useful information
such as URLs, identifications numbers, names and descriptions or some dates.

% TODO
% What do rows of data matrix contain? (one paragraph)

% Metadata Table

% TODO - Recompilar metadata
\includepdf[pages=-,landscape=true, addtotoc={
1,subsection,1,Metadata Table,metadata_table
}
]{../metadata_table/metadata}

% TODO
% Final scope of the study with inclusion and exclusion criteria for both rows
% and columns (max half a page)
\subsection{Final scope of study}

After considering what variables were appropriate for our study, we decided to keep
those shown in the metadata table in \cref{metadata_table}. We chose these because
we believe they provide real insight into the question we are posing in this study,
in which we aim to find what factors affect most the obtention of positive reviews
and high ratings. Those are variables that describe the nature of the host or
the listing in an abstract manner, allowing us to analyze the listings adequately.

We discarded the rest
of the variables during preprocessing, either because they were duplicated or
irrelevant. Furthermore, we extrapolated a few variables and created
two new ones, this process is explained in higher detail in the preprocessing
section. This resulted in a cleansed data set that contains 30 variables:
20 numerical, 7 categorical and 4 boolean.

%% vim: spell spelllang=en:
%! TEX root = **/00-main.tex

% Complete Data Mining process performed (one page, including a workflow).

\section{Data mining process}%
\label{sec:data_mining_process}

%% vim: spell spelllang=en:
%! TEX root = **/00-main.tex

% Detailed description of Preprocessing and data preparation. Please be sure to
% justify all decisions made.

\section{Preprocessing}%
\label{sec:preprocessing}

%% vim: spell spelllang=en:
%! TEX root = **/00-main.tex

% Basic statistical descriptive analysis

\section{Basic statistical descriptive analysis}%
\label{sec:basic_statistical_descriptive_analysis}

% Univariate for all the variables included in the study (half a page per variable)
\subsection{Univariate analysis}%
\label{sub:univariate_analysis}

% Bivariate when relevant (half a page per pair of variables)
\subsection{Bivariate analysis}%
\label{sub:bivariate_analysis}

% When required, please include descriptives before and after preprocessing

% Conclude the section with one paragraph describing how is your data

%% vim: spell spelllang=en:
%! TEX root = **/00-main.tex

% PCA analysis for numerical variables:

\section{PCA analysis for numerical variables}%
\label{sec:pca_analysis_for_numerical_variables}

% Scree plot. Specify how many principal components are selected
\subsection{Scree plot}%
\label{sub:scree_plot}


\begin{figure}[H]
    \centering
    \includegraphics[width=0.7\linewidth]{pca_fact-screeplot} % PCA-inertia_cum
    \caption{PCA inertia}%
    \label{fig:pca_inertia}
\end{figure}

\begin{figure}[H]
    \centering
    \includegraphics[width=0.7\linewidth]{PCA-inertia_cum}
    \caption{PCA accumulated inertia}%
    \label{fig:pca_inertia_cum}
\end{figure}

\vspace{-1em}
We used the inertia plots to decide the number of factorial axis to analyse. In
\cref{fig:pca_inertia} we can conclude the 4 first axis represent a much
larger amount of variance compared to the others. When looking at
\cref{fig:pca_inertia_cum} we can see that the first 4 axis already contain 63\%
of the variance. If we analyse further with the elbow method we can clearly see
that the slop starts to decrease at around the 4\ts{th} axis. Therefore we decided to
only further analyse the first 4 factorial axis.

% 20.24968405 17.88042469 13.04514781 12.39547495
% 20.24968  38.13011  51.17526  63.57073     %  69.09301  74.04671  78.51992  82.73880


% Factorial map visualisation:
\subsection{Factorial map visualisation}%
\label{sub:factorial_map_visualisation}

% #2 -> caption #3 -> file, #1 -> page
\newcommand{\factorialmap}[2]{
    \begin{figure}[H]
        \centering
        %\includegraphics[trim=3cm 0 3cm 0, clip, width=0.65\linewidth]{plane_#1_#2-var}
        %\parbox[t]{0.24\linewidth}{lorem ipsum dolor sit amet adescipng susisf jsdasj fsdlk ks aksjldsdf jkls}
        \includegraphics[width=\textwidth]{pca_fact-plane_#1_#2-var}
        \caption{PCA plane #1 vs #2}%
        \label{fig:plane_#1-#2}
    \end{figure}
}

\newcommand{\contrib}[2]{
    \begin{figure}[H]
        \centering
        \includegraphics[width=0.85\linewidth]{pca_fact-plane_#1_#2-contrib}
        \caption{PCA variable contributions of plane #1 vs #2}%
        \label{fig:contrib_plane_#1-#2}
    \end{figure}
}

\newcommand{\categorica}[4]{
    \begin{figure}[H]
        \centering
        \includegraphics[width=0.85\linewidth]{pca_fact-#3-plane_#1_#2}
        \caption{PCA variable contributions of #4 in plane #1 vs #2}%
        \label{fig:cat-#3-plane-#1-#2}
    \end{figure}
}

%TODO

\factorialmap{1}{2}

In \cref{fig:plane_1-2} we are showing the factorial maps that represent the most variance among all our factorial axis. When looking at the numerical variables represented we can see that all the variables
related to reviews tend to have similar arrows, this makes sense 
because in the bivariate analysis we concluded that they were correlated.
Looking at \cref{fig:contrib_plane_1-2} we can see that the variables
with higher contribution are the ones about review scores and availability. As all review score variables have a small angle to the first factorial axis we can begin to speculate that this axis is 
mostly affected by these variables. Other variables like availability
and accommodates seem to contribute mostly to the second factorial axis, however these angles aren't as tight so they probably affect the first axis as well.


\factorialmap{1}{3}
In \cref{fig:plane_1-3} we can see that it seems to hold true that 
the review score variables affects axis 1 the most. It is pretty clear that the latent variables associated to the 
first factorial map might be the reviews score variables. When looking at accommodates, beds and 
bedrooms we see that they have similar arrows, that again makes sense because there is some 
correlation between them. These 3 seem to contribute much more to the third factorial axis.

\factorialmap{1}{4}


\factorialmap{2}{3}


\factorialmap{2}{4}
What we can extract when analysing \cref{fig:plane_1-3} is that the availability variables have 
a relatively high contribution and form a very small angle with the second factorial axis.

\factorialmap{3}{4}
When looking at \cref{fig:plane_3-4} we can see that the variable arrows seem to group in 3 different groups: the ones regarding number of reviews, the ones about availability and others related to accommodates and bedroom. If we analyse \cref{fig:contrib_plane_3-4} we can see that the ones that contribute the most are bedrooms, accommodates 
and reviews per month. It is hard to interpret the latent variables associated using this particular figure because the most contributing variables have large angles to both axis.


\begin{landscape}

\contrib{1}{2}
\contrib{3}{4}

\categorica{1}{2}{room_type}{room type}
\categorica{1}{3}{room_type}{room type}
\categorica{1}{4}{room_type}{room type}

\subsection{Conclusion}%
What we can conclude after analysing how the variables are distributed along each factorial axis is 
that our variables generally distribute in 4 groups: the ones about availability, 
the ones about review scores, the ones regarding number of reviews and finally beds and accommodation.

After studying each factorial axis in depth we can concluded that the latent variables associated to 
the first factorial map are the ones that talk about review scores. We reached these conclusion after
seeing that in all PCA plots with with the first factorial axis, these variables were top contributors
and their angles to the first axis were really small. The relation seems to be direct as in all plots 
they appear in the positive first axis.

\end{landscape}

% (Be sure you use a single landscape pager for each single map in order to
% guarantee visibility of materials to the readers)

% For each factorial map provide:

% - Individuals projections

% - Common projection of numerical variables and modalities of qualitative
% - variables (take care to use correct color codes as explained along the course)

% - Interpretation of relationships among variables observed. When possible,
% - interpret the latent variable associated with the principal axis

% - Conclusions

% Note: All factorial maps must be placed in a single landscape page that makes
% it visible

%% vim: spell spelllang=en:
%! TEX root = **/00-main.tex

% Hierarchical Clustering on original data:

\section{Hierarchical Clustering}%
\label{sec:hierarchical_clustering}

% TODO:

% Precise description of the data used (which variables have not been included
% in the analysis, if any, whenever you are using a CURE strategy, provide
% details about eventual sampling performed on data, etc)

% Clustering method used, metrics and aggregation criteria used (Ward’s method
% is recommended embedded or not in a CURE strategy whenever scaling to big data
% is required, for messy data Gower dissimilarity coefficient to the square is
% recommended )

We decided to investigate three different hierarchical clustering methods to
identify which one worked best with our data set: Ward's method, and 
Hierarchical Clustering on Principal Components.

\begin{landscape}

\subsection{Dendogram}%
\label{sub:dendogram}

% Resulting Dendrogram (of the total dataset or the sample). USE A SINGLE PAGE
% for it

\begin{figure}[H]
    \centering
    \includegraphics[width=0.8\linewidth]{dendo}
    \caption{Cluster Dendogram}%
    \label{fig:dendogram}
\end{figure}

\end{landscape}

\begin{table}[H]
\end{table}

% Discuss about how to get the final number of clusters

\subsection{Description of clusters}%
\label{sub:description_of_clusters}



% Table with a description of the clusters size

%% vim: spell spelllang=en:
%! TEX root = **/00-main.tex

\section{Profiling of clusters}%
\label{sec:profiling_of_clusters}

\newcommand{\profiling}[3]{
\begin{figure}[H]
    \centering
    \includegraphics[width=0.7\textwidth]{prof-c3-#1-#2}
    \caption{#3}%
    \label{fig:#1-#2}
\end{figure}
}

\profiling{room_type}{percent}{room type bar}
\profiling{room_type}{side}{room type bar}
\profiling{room_type}{stack}{room type bar}

% Profiling of clusters: Use class variable as a response variable to analyze
% conditional distributions of variables to clusters and eventual statistical
% tests to assess which variables are significant in each cluster. Detect
% commonalities of each cluster and differences between clusters. What is
% intrinsic of each cluster? What distinguishes clusters among them?

% Profiling graphs, CPGs, multiple boxplots, bivariate barplots, descriptive by
% groups, etc...

% For selected relevant variables, you can also add specific profiling tests to
% complete clusters interpretation

% Synthesize the result of the classes’ interpretation process into a set of
% templates characterizing the clusters, one template per cluster

%% vim: spell spelllang=en:
%! TEX root = **/00-main.tex

% Global discussion and general conclusions of the whole work. Analyze
% coincidences and divergences between ACP, AMC, Clustering

\section{Conclusions}%
\label{sec:conclusions}

One of the aims of the projects was to explain what variables have an effect on a customer leaving a good or bad review score. After doing the univariate analysis we found out that reviews were pretty polarized.  In fact, most of the scores were pretty high, but there was a small percentage of really bad ones. When analyzing the clusters created by HCPC, it turned out that we accumulated the really bad reviewed listings in cluster 3, while the other three contained mostly excellent scores. This made it hard for us to determine the characteristics of a good review. Because of that, we thought that it would be appropriate to study the characteristics of the listings that had bad review scores, since they were more distinguished and less broad. We concluded that most bad reviews tend to be posted on listings not owned by a superhost, that usually has a small number of preview reviews. We also found out that listings for shared rooms where more likely to end up having a bad review.

We also managed to conclude that cluster 4 contained mostly bigger and more expensive listings that could accommodate more guests. This is interesting but it doesn't help us answer any of the questions we had proposed at the beginning of the project.
In the opposite side of the spectrum,  cluster number 1 contains mostly the listings for the smaller apartments and private rooms, with good reviews and cheaper prices. This kind of listing seems to be the most popular one among Airbnb in Barcelona, which in our experience completely makes sense.

We have analysed further clusters 1,2 and 4 as to find any major difference between them. Although some anecdotal facts have been found we don't think there's enough evidence to derive any further substantial differences.
%% vim: spell spelllang=en:
%! TEX root = **/00-main.tex

% Working plan, including (please be sure you include the working plan at the end
% of the document, not at the beginning)

% Initial and final Gantt

% vim: spell spelllang=en:
%! TEX root = **/00-main.tex

\newgeometry{top=0.5cm, bottom=0.5cm, left=0.5cm, right=0.5cm}
\begin{landscape}

\section{Working plan}%
\label{sec:working_plan}

\null
\vspace{1em}
{
\centering
\subsection{Initial Gantt}%
\label{sub:ini_gantt}
\vspace{2em}
\par}

\begin{center}
\begin{ganttchart}[
vgrid={*{6}{draw=none}, dotted},
x unit=.75cm,
y unit title=1cm,
y unit chart=1cm,
    time slot format=isodate
    ]{2020-09-15}{2020-10-07}
\gantttitlecalendar{month=name, day}
\ganttnewline

\ganttgroup{Motivation and Description}{2020-09-16}{2020-09-23}
\ganttnewline

\ganttgroup{Data source presentation}{2020-09-16}{2020-09-23}
\ganttnewline

\ganttbar{Description of data source}{2020-09-16}{2020-09-23}
\ganttnewline

\ganttgroup{Formal description of data structure}{2020-09-23}{2020-09-30}
\ganttnewline
\ganttbar{Metadata Table}{2020-09-23}{2020-09-25}
\ganttnewline
\ganttbar{Scope of study}{2020-09-25}{2020-09-30}
\ganttnewline

\ganttgroup{Description of preprocessing}{2020-09-23}{2020-09-27}
\ganttnewline

\ganttgroup{Basic statistical descriptive analysis}{2020-09-27}{2020-10-07}
\ganttnewline
\ganttbar{Univariate analysis}{2020-09-27}{2020-09-30}
\ganttnewline
\ganttbar{Bivariate analysis}{2020-10-01}{2020-10-04}
\ganttnewline
\ganttbar{Conclusion describing data}{2020-10-04}{2020-10-07}
\ganttnewline
\end{ganttchart}
\end{center}

\pagebreak
\null
\vspace{3.5em}
\begin{center}
\begin{ganttchart}[
vgrid={*{6}{draw=none}, dotted},
x unit=.85cm,
y unit title=1cm,
y unit chart=1cm,
    time slot format=isodate
    ]{2020-10-07}{2020-10-28}
\gantttitlecalendar{month=name, day}
\ganttnewline

\ganttgroup{PCA analysis for numerical variables}{2020-10-07}{2020-10-14}
\ganttnewline
\ganttbar{Scree plot}{2020-10-07}{2020-10-11}
\ganttnewline
\ganttbar{Factorial map visualization}{2020-10-11}{2020-10-14}
\ganttnewline

\ganttgroup{Hierarchical Clustering}{2020-10-14}{2020-10-21}
\ganttnewline
\ganttbar{Clustering script}{2020-10-14}{2020-10-19}
\ganttnewline
\ganttbar{Description of data selected}{2020-10-19}{2020-10-21}
\ganttnewline
\ganttbar{Clustering method and metrics used}{2020-10-19}{2020-10-21}
\ganttnewline
\ganttbar{Resulting Dendogram}{2020-10-19}{2020-10-21}
\ganttnewline
\ganttbar{Discussion about number of clusters}{2020-10-19}{2020-10-21}
\ganttnewline
\ganttbar{Table with description of cluster size}{2020-10-19}{2020-10-21}
\ganttnewline

\ganttgroup{Profiling of clusters}{2020-10-21}{2020-10-28}
\ganttnewline
\ganttbar{Profiling graphs}{2020-10-21}{2020-10-28}
\ganttnewline

\end{ganttchart}
\end{center}
\end{landscape}

\restoregeometry

% vim: spell spelllang=en:
%! TEX root = **/00-main.tex

\newgeometry{top=0.5cm, bottom=0.5cm, left=0.5cm, right=0.5cm}
\begin{landscape}

\null
\vspace{1em}
{
\centering
\subsection{Final Gantt}%
\label{sub:fin_gantt}
\vspace{2em}
\par}

\begin{center}
\begin{ganttchart}[
vgrid={*{6}{draw=none}, dotted},
x unit=.75cm,
y unit title=1cm,
y unit chart=1cm,
    time slot format=isodate
    ]{2020-09-15}{2020-10-09}
\gantttitlecalendar{month=name, day}
\ganttnewline

\ganttgroup{Motivation and Description}{2020-09-16}{2020-09-21}
\ganttnewline

\ganttgroup{Data source presentation}{2020-09-16}{2020-09-22}
\ganttnewline

\ganttbar{Description of data source}{2020-09-16}{2020-09-22}
\ganttnewline

\ganttgroup{Formal description of data structure}{2020-09-23}{2020-09-30}
\ganttnewline
\ganttbar{Metadata Table}{2020-09-23}{2020-09-25}
\ganttnewline
\ganttbar{Scope of study}{2020-09-25}{2020-09-30}
\ganttnewline

\ganttgroup{Description of preprocessing}{2020-09-23}{2020-09-28}
\ganttnewline

\ganttgroup{Basic statistical descriptive analysis}{2020-09-27}{2020-10-09}
\ganttnewline
\ganttbar{Univariate analysis}{2020-09-27}{2020-10-02}
\ganttnewline
\ganttbar{Bivariate analysis}{2020-10-03}{2020-10-08}
\ganttnewline
\ganttbar{Conclusion describing data}{2020-10-08}{2020-10-09}
\ganttnewline
\end{ganttchart}
\end{center}

\pagebreak
\null
\vspace{3.5em}
\begin{center}
\begin{ganttchart}[
vgrid={*{6}{draw=none}, dotted},
x unit=.85cm,
y unit title=1cm,
y unit chart=1cm,
    time slot format=isodate
    ]{2020-10-07}{2020-10-28}
\gantttitlecalendar{month=name, day}
\ganttnewline

\ganttgroup{PCA analysis for numerical variables}{2020-10-07}{2020-10-12}
\ganttnewline
\ganttbar{Scree plot}{2020-10-07}{2020-10-10}
\ganttnewline
\ganttbar{Factorial map visualization}{2020-10-10}{2020-10-12}
\ganttnewline

\ganttgroup{Hierarchical Clustering}{2020-10-12}{2020-10-20}
\ganttnewline
\ganttbar{Clustering script}{2020-10-12}{2020-10-16}
\ganttnewline
\ganttbar{Description of data selected}{2020-10-17}{2020-10-18}
\ganttnewline
\ganttbar{Clustering method and metrics used}{2020-10-17}{2020-10-18}
\ganttnewline
\ganttbar{Resulting Dendogram}{2020-10-17}{2020-10-19}
\ganttnewline
\ganttbar{Discussion about number of clusters}{2020-10-19}{2020-10-20}
\ganttnewline
\ganttbar{Table with description of cluster size}{2020-10-19}{2020-10-20}
\ganttnewline

\ganttgroup{Profiling of clusters}{2020-10-20}{2020-10-26}
\ganttnewline
\ganttbar{Profiling graphs}{2020-10-20}{2020-10-26}
\ganttnewline

\end{ganttchart}
\end{center}
\end{landscape}

\restoregeometry
 % TODO

% Final tasks assignment grid

% vim: spell spelllang=en:
%! TEX root = **/00-main.tex

\subsection{Final task assignment}%
\label{sub:division_of_tasks}

\newcommand*\rot{\rotatebox{90}}
\newcommand*\X{\ding{56}}
\newcommand*\x{{\color{gray}\ding{55}}}
\begin{table}[H]
\centering
\begin{tabular}{@{}l|c|c|c|c@{}}
             & \rot{Aleix Boné} & \rot{Eduard Bosch} & \rot{David Gili} & \rot{Albert Mercadé} \\
\toprule
\textbf{Coordination}                           &    &    &\X  &    \\ \midrule
\textbf{Motivation and Description}             & \x &    &    & \X \\ \midrule
\textbf{Data source presentation}               &    &    &    &    \\
Description of data source                      &    & \x &    & \X \\ \midrule
\textbf{Formal description of data structure}   &    &    &    &    \\
Metadata Table                                  &\x  &\x  & \X &    \\
Scope of study                                  &\X  &\x  &    &    \\ \midrule
\textbf{Description of prepossessing}           &    &\X  & \x & \x \\ \midrule
\textbf{Basic statistical descriptive analysis} &    &    &    &    \\
Univariate analysis                             & \x & \x &    & \X \\
Bivariate analysis                              & \X &    & \x &    \\
Conclusion describing data                      &\X  & \x &    &\x  \\ \midrule
\textbf{PCA analysis for numerical variables}   &    &    &    &    \\
Scree plot                                      &    &    &\x  & \X \\
Factorial map visualization                     &    & \X & \x &    \\ \midrule
\textbf{Hierarchical Clustering}                &    &    &    &    \\
Clustering script                               &\x  & \X &    & \x \\
Description of data used                        &    &    & \X &\x  \\
Clustering method and metrics used              & \X &\x  &    &    \\
Resulting Dendogram                             &\X  &    &\x  &\x  \\
Discussion about number of clusters             &    &\X  &    &    \\
Table with description of cluster size          &\x  &    &\X  &    \\ \midrule
\textbf{Profiling of clusters}                  &    &    &    &    \\
Profiling graphs                                &\x  &    &    &\X  \\
\end{tabular}
\end{table}


% TODO
% Critical discussion about deviances of final scheduling with respect to the
% originally designed one and discussion about risks avoided by the initial
% contention plan and unexpected risks appeared during project.

%
%\printbibliography
%
%\appendix
%
%%% vim: spell spelllang=en:
%! TEX root = **/00-main.tex

% R Scripts (only if they have not been embedded along the explanations of the
% work in previous chapters)

\section{R Scripts}%
\label{sec:r_scripts}

%% vim: spell spelllang=en:
%! TEX root = **/00-main.tex

% Additional plots

\section{Additional plots}%
\label{sec:additional_plots}

% TODO


\end{document}
